\documentclass[12pt,a4paper]{article}

\usepackage{mystyle} % mit dieser Zeile werden die Pakete in das Dokument eingelesen, das Dokument mystyle.sty muss dazu im directory vorhanden sein

\newcommand{\Title}{The Information Content of VIX Volatility}
%in Comparison to Historic Volatility} 
\newcommand{\Arbeit}{Humboldt Research Project Project}
\newcommand{\Seminar}{Modul 11251 Forschungsprojekt}
\newcommand{\Name}{Sophia Charlotte Gläser}
\newcommand{\MatrikelNummer}{15202284}
\newcommand{\Programme} {Corporate Management and Economics}
\newcommand{\Date}{31.01.2019}
\newcommand{\Semester}{Fall Semester 2018}
\newcommand{\Pruefer}{Prof. Dr. Franziska Peter}
\newcommand{\Chair}{Chair of Empirical Finance and Econometrics}

	
\addbibresource{./bib/bibliography.bib} % das entsprechende File muss ebenfalls im entsprechenden Ordner vorhanden sein


\usepackage{acro}
\DeclareAcronym{BS}{short = BS, long = Black-and-Scholes-Merton Model}
\DeclareAcronym{CBOE}{short = CBOE, long = Chicago Board of Options Exchange}
\DeclareAcronym{SPX}{short = SPX, long = S\&P 500 index}
\DeclareAcronym{H1}{short = H1, long = Hypothesis 1, class = hypo}
\DeclareAcronym{H2}{short = H2, long = Hypothesis 2, class = hypo}
\DeclareAcronym{H3}{short = H3, long = Hypothesis 3, class = hypo}
\DeclareAcronym{H4}{short = H4, long = Hypothesis 4, class = hypo}
\DeclareAcronym{VIX}{short = VIX, long = Volatility index}

\begin{document}


	\begin{centering}
\Large \textbf{Zeppelin Universität}\\
\Large \Chair \\
\vfill
\LARGE \textbf{\Title} \\
\vfill
\LARGE \Arbeit\\
\vfill
\begin{small}
\begin{doublespace}
\begin{tabbing}
	Student numberrrrrrrr \=\kill
	Name:\>\Name\\
	Student number:\>\MatrikelNummer\\
	Programme:\>\Programme\\
	Term:\>\Semester\\
	Examinor:\>\Pruefer\\
	Due date:\>\Date
	\end{tabbing}
\end{doublespace}
\end{small}
\end{centering}\vspace{1cm}

% Deutsche Version
% \begin{centering}
% \Large \textbf{Zeppelin Universität}\\
% \Large \Chair \\
% \vfill
% \LARGE \textbf{\Title} \\
% \vfill
% \LARGE \Arbeit\\ %Bachelorarbeit
% \Large in \\
% \LARGE \Seminar\\
% \vfill
% \begin{small}
% \begin{doublespace}
%	\begin{tabbing}
%	Immatrikulationsnumerrrrrr \=\kill
%	Bearbeitet von:\>\Name\\
%	Immatrikulationsnummer:\>\MatrikelNummer\\
%	Studiengang:\>\Programme\\
%	Semester:\>\Semester\\
%	Prüfer:\>\Pruefer\\
%	Abgabedatum:\>\Date
%	\end{tabbing}
% \end{doublespace}
% \end{small}
% \end{centering}\vspace{1cm}
	\newpage
	
	\pagenumbering{Roman}
	\tableofcontents
	\newpage
	\listoftables
	\listoffigures
	\printacronyms[exclude-classes=hypo]
	
	\newpage
	
	\pagenumbering{arabic}
	\setcounter{page}{1}
	
	\begin{abstract}
		%!TEX root = ../Main.tex 

This paper investigates the information content of model-free implied volatility for daily realized volatility of the S\&P 500, using the \ac{VIX} from the Chicago Board of Options Exchange. The data set contains daily volatilities from 2000 to 2017. In contrast to the Black-and-Scholes implied volatility, the VIX is not based on any specific option pricing model, therefore it provides a direct test of market efficiency and does not suffer from the joint hypothesis problem. This paper examines information content of the VIX in comparison to historic volatility. The approach from an HAR-RV model as described by \textcite{corsi2009} is used, thus including not only daily, but also weekly and monthly historic volatility in the encompassing regression analysis. The results show, that the VIX provides additional information compared to historic volatility, but is not able to subsume all the information contained in historic volatilities, which partly contradicts previous research. This results are robust to serial correlation and alternative sampling methods. 
	\end{abstract}

	%!TEX root = ../Main.tex

\section{Introduction: The Importance of Volatility Estimation}\label{sec:1Intro}

Financial market volatility is of high interest for the financial sector. Asset return volatility is for example key input to the pricing of financial instruments like derivatives, or to risk measures such as the Value at Risk. Moreover they give information on the risk-return trade-off, which is a central question in portfolio allocation and managerial decision making. \\
As however volatility is not directly observable, it has to be estimated. Seeing its importance, it is not astonishing that during the last years, considerable research has been devoted to the question, how volatility can be estimated and predicted. Two prominent approaches that have been commonly used are time series models, like the ARCH or stochastic volatility models, or implied volatility models, like the \gls{BS} implied volatility models. Whereas time series models rely on historic data, implied volatility models use option price data\footnote{There are, of course, various other methods for volatility estimation and forecasting, such as various nonparametric methods or neural networks based models. However, they shall not be discussed here, for an encompassing overview of volatility estimation and forecasting can be found in \textcite{jiang2003}.}.\\
Out of these two, there has been a growing interest in implied volatility during the the recent years. Since options are contracts giving the holder the right to buy or sell an underlying asset at a specified date in the future, they are said to have a ``forward-looking nature'', meaning that they are supposed to be highly related to the market's expectation about the future volatility of the underlying asset over the remaining life of the option. Therefore, if market agents are rational (meaning that the market uses all available information to form it's expectations about future price movements and volatiltiy) markets are efficient and the model pricing the option is specified correctly, the volatility implied from option prices should be an unbiased and efficient estimator of future realized volatility \parencite{bakanova2010}. Moreover, it should be the best possible forecast possible, given the current information \parencite{christensen2002}. \\
For some time, one popular approach for estimating implied volatility has been the \gls{BS} implied volatility. The \gls{BS} model is an option pricing model, using volatility as an input factor. By using observed option prices as the input and solving for volatility, it is possible to obtain a volatility measure that is widely believed to be ``informationally superior to the historic volatiltiy of the underlying asset'' \parencite[p.1305]{jiang2003}. Early studies found this \gls{BS} implied volatility to be a biased forecast of realized volatility, not containing significantly more information than historic volatility. More recent studies however rejected thesse findings and presented evidence that there is indeed additional information contained in option prices \parencite{jiang2003}. A reason for this discrepancy could be that early studies did not consider several data and methodological problems, such as long enough time series, a possible regime shift around the crash in 1987 and the use of non-overlapping samples \parencite{jiang2003}. \citeauthor{christensen1998} for example took this into account and found that implied volatility outperforms historic volatility. All in all, this new insights provided evidence against the inefficiency of implied volatility.\\
However, even though \gls{BS} implied volatility is found to be the overall more efficient forecast of realized volatility compared to historic volatility, the \gls{BS} implied volatility has some specification problems. Firstly, \gls{BS} implied volatility focuses on at-the-money options. The advantage is, that at-the-money options are the once most actively traded and thus the most liquid ones. However this focus fails to include information contained in other options. Moreover, volatility estimation with the \gls{BS} model includes the same assumptions as are made in the \gls{BS} model itself. Thus, tests based on the \gls{BS} equation are joined tests of market efficiency (as market efficiency has to be assumped to use option prices for volatility estimation, as mentioned above) and the \gls{BS} model, and therefore suffer from model misspecification errors \parencite{jiang2003}. \\
That is why during the last years, implied volatility indexes which are not based on a pricing assumption have gained popularity. One of these model-free implied volatility indexes is the VIX from \gls{CBOE}.






	\newpage
	%!TEX root = ../Main.tex

\section{Selected volatility concepts and models of volatility measurement}
This section presents first some stylized facts of financial data, and gives an introduction to the different ways to estimate volatility. By pointing out the advantages and disadvantages of the concepts and models and their fit to the stylized facts, the HAR-RV approach shall be motivated. 


\subsection{The Return Process and Stylized Facts of Financial Data}
As mentioned in the introduction, the challenge when measuring volatility is, that stock return volatility is not directly observable \parencite{tsay2005}. This problem evolves from the fact that we can only observe one realization of the underlying data generating process, and even though stocks are traded and thus have market prices which could be used for volatility measurement, there is no continuous data available and even for high-frequency data and extremely liquid markets microstructure effects and noise prevent getting close to a continuous sample path. It is thus only possible to estimate avearages of discrete volatility for a given period of time. \parencite{andersen2001}.\\
There are however several approaches that should be introduced here. To start with, the definition of the simple gross return is
\begin{align}\label{eq:return}
1+ R_{t} = \frac{P_{t}}{P_{t-1}} 
\end{align}
In the continuous-time setting, continuously compounded returns are used, which are given by
\begin{align}\label{eq:log-return}
r_{t} = ln(1 + R_{t}) = ln (\frac{P_{t}}{P_{t}} + \frac{P_{t-1} - P_{t}}{P_{t}}) = 
ln \frac{P_{t}}{P_{t-1}} = p_{t} - p_{t-1} \ 
with\  p_{t} = ln(P_{t})
\end{align}
When observed over time, this asset returns show some distributional properties, often referred to as stylized facts of asset returns. Observed by many authors, only a few shall be mentioned here. \citeauthor{corsi2009} for example mentions particularly the very strong persistence of autocorrelation of the square and absolute returns, which regularly poses challenges to econometric models. Moreover return probability density functions are often leptocurtic with fat tails. As the time scale increases, the return distribution slowly converge to the normal distribution, but before convergence, the return distribution has different shapes depending on the time scale. Financial data also show evidence of scaling as described firstly by Mandelbrot, which is connected to the idea that patterns appear in different times, or that the distribution for returns has similar functional forms for various choices of the time interval. \\
Moreover \citeauthor{andersen2001} mentions that first, even though raw returns have a leptocurtic distribution, the returns standardized by realized volatility are approximately Gaussian. Second, the distribution of realized volatility of returns itself is right skewed, the one of the logarithms of realized volatility however are also approximately Gaussian. Third, the long-run dynamics of realized logarithmic volatilities are well approximated by a fractionally-integrated long-memory process. Other authors who mention stylized facts: \parencite{jiang2003}.

\subsection{Concepts and Models using Historic Volatility}
\subsubsection{Volatility Concept and Non-parametric ex-post Volatility Measurement - Realized Volatility}
By definition ``volatility seeks to capture the strength of the (unexpected) return variation over a given period of time'' \parencite[p.7]{andersen2001}. However, there are multiple concepts and definitions of asset volatility. According to \citeauthor{andersen2001} the concepts can be grouped in (i) the \emph{notional volatility} corresponding to the ex-post sample-path return variability over a fixed time interval, (ii) the ex-ante \emph{expected volatility} over a fixed time interval or the (iii) the \emph{instantaneous volatility} corresponding to the strength of the volatility process at a point in time.
For this paper, given the dataset of actual return observations, one can compute the ex-post realized volatility.\\
It can be shown, that under some assumptions, realized volatility as the sum of squared high frequency returns, can be used to approximate the quadratic variation process which is the variation in a continuous time setting. This approach mainly building on the work of \citeauthor{andersen2001} und [noch jemanden finden] shall only be briefly introduced here. \\
To begin with, it should be assumed that we have a continuous-time no-arbitrage setting. As return volatility aims to capture the strength of the unexpected return variation, one needs to define the component of a price change as opposed to an expected price movement. In discrete time this can be done by specifying the conditional mean return using for example an asset pricing model. In the continuous time setting however it requires the decomposition of the return process in an expected and innovation component. \citeauthor{andersen2001} show, that under certain assumptions the log-price process must constitute a semi-martingale process, which allows for the decomposition of the instantaneous return process into an expected return component, and a martingale innovation. They show furthermore, that one can refer to the quadratic variation process of this martingale component as a volatility measure, as the quadratic variation process represents the (cumulative) realized sample path variability of the martingale over any fixed time interval. To be precise, they define \emph{notional volatility} as the increment to the quadratic variation for the return series, measured ex-post.\\
Assuming that the mean of the return process is zero, taking the expected value of the notional volatility and extending this concept slightly, one gets the \emph{realized volatility}, defined over the $[t-h,t], 0 < h \leq t \leq T$ time interval as
\begin{align}\label{eq:RV-andersen}
v^2(t,h;n) = \sum_{i=1}^{n} r(t-h+(i/n) \times h,h/n)^2
\end{align}
\citeauthor{andersen2001} show not only that the realized volatility is an \emph{unbiased} estimator of ex-ante expected volatility, or at least approximately unbiased when relaxing the zero mean assumption and taking a high sample frequency (their proposition 4). Moreover, \citeauthor{andersen2001} show that the realized volatility is a \emph{consistent} nonparametric measure of the notional volatility for increasingly finely sampled returns over any fixed length interval (their proposition 5). So in summary, the increment to the quadratic return variation and thus past volatility can be consistently and well approximated through the accumulation of high-frequency squared returns.\\

For the purpose of this paper it shall only be said, that quadratic variation is... .  


\subsubsection{Volatility Model - HAR-RV Model}
Having introduced the concept of notional volatility and it's approximation by realized volatility, we now turn to volatility modelling/measurement. To measure volatility, one can separate between parametric and non-parametric methods. Whereas parametric methods try to measure the expected volatility making different assumptions about both the functional form and the variables in the information set available, non-parametric methods try to quantify notional volatility directly. The realized volatility is an example for a non-parametric methods. However, to forecast or estimate volatility ex-ante, this paper will refer to one type of the parametric methods, termed the HAR-RV model.\\
As mentioned in the section above, the logreturn process can be decomposed in a predictable and finite variation process, and a local martingale. \citeauthor{corsi2009}, does so by assuming the standard continuous time diffusion process:
\begin{align}\label{eq:return-process-corsi}
dp_{t} = \mu (d) dt + \sigma_{t} dW_{t}
\end{align}
with $p(t)$ being the logarithm of the instantaneous price, $\mu (t)$ a cadlág finite variation process, $W (t)$ a standard Brownian motion, and $\sigma (t)$ a stochastic process independent of $W_{p,t}$.\\
As in \citeauthor{andersen2001} they approximate the instantaneous/notional variance with the sum of squared returns, but term not this variance as volatility, but it's square root. This terminology should also be used for the remainder of this paper, with the realized volatility for one trading day being then:
\begin{align}
RV_{t}^{(d)} = \sqrt{\sum_{j=0}^{M-1} r^{2}_t-j \times \Delta}
\end{align}
with $\Delta = 1d/M$ being the sampling frequency and $r^{2}_t-j \times \Delta$ defined as the continuously compounded $Delta$-frequency returns. \\
Combining this notion of volatility with the Heterogeneous Market Hypothesis by \citeauthor{mueller1993}. This 



Also \citeauthor{andersen2003} point out the advantage of using high-frequency returns is not only that they help predicting again high-frequency returns, but also that they contain information for longer horizons, such as monthly or quarterly. 


\subsection{Implied volatility}
\subsubsection{The General Idea of Implied Volatility}
\citeauthor{andersen2001} define \emph{implied volatility} as consisting of a parametric volatility model for returns, accompanied by an asset pricing model and an augmented information set, including also option prices. Having the derivative prices, it is possible to extract a value for the expected volatility, by inverting the theoretical asset pricing model. It is however important to note, that all of these procedures depend on the assumptions that are made in the asset pricing model \parencite{andersen2001}. 


\begin{itemize}\itemsep0pt
\item explain basic idea of BS implied volatility
\item advantages of BS implied volatility: forward-looking nature of option prices
\item disadvantages of BS implied volatility: joint hypothesis problem due to underlying  pricing assumption (is a joint test of market efficiency and underlying pricing assumption), use only at-the-money options and fail to incorporate information,..
\end{itemize}

\textcolor{gray}{
Disadvantages of Black and Scholes: Black and Scholes uses only at-the-money option and thus fails to incorporate information \parencite{jiang2003}.
Black and Scholes are joint tests of market efficiency and the B-S model, thus studies are subject to model misspecification errors \parencite{jiang2003}.}

\subsubsection{VIX and Model-Free Implied Volatility}
\begin{itemize}\itemsep0pt
\item explain basic idea of model-free implied volatility
\item advantages of model-free implied volatility: solved joint hypothesis problem (direct test of market efficiency), can incorporate not only at-the-money options,..
\item the VIX as the model-free implied volatility estimate from the Cboe 
\end{itemize}

\textcolor{gray}{
Primilary described and derived by \citeauthor{britten2000}. Instead of being based on a specific option pricing model, it is derived entirely from no-arbitrage conditions. After that some papers did various corrections, such as \citeauthor{jiang2003} extended the model so that is not derived under diffusion assumptions and generalized it to processes including random jumps. Two advantage of the model-free option implied volatility, are firstly that it has no pricing assumption and thus constitutes a direct test of the option market's informational efficiency, and not a joined test of market efficiency and an assumed option pricing model. Secondly it incorporates information from options across different strike prices. }
 
	\newpage
	%!TEX root = ../Main.tex

\section{Methodology and Data}

\subsection{Data and Calculation of Input Factors}
The data used in this study are from several sources. The realized variance of the S\&P 500 .. . Then write about what data adjustments were made, including the ones already done with oxford realized library.

Also \citeauthor{andersen2003} point out the advantage of using high-frequency returns is not only that they help predicting again high-frequency returns, but also that they contain information for longer horizons, such as monthly or quarterly. 
Graphics

VIX is the daily implied volatility index, calculated from the annualized volatility as VIX/$\sqrt{252}$ as in \textcite{blair2000} and \textcite{whaley2008}. \\

For an analysis of the historic values of the VIX, see \textcite{whaley2008} (chapter: historix, normal range). 


\begin{figure}[!htbp]
\includegraphics[width=16cm, height=8cm]{pictures/SPandViX.png}
\end{figure}

\begin{figure}[!htbp]
\includegraphics[width=16cm, height=8cm]{pictures/SPandVolandViX.png}
\end{figure}

Measure for daily return variability should be realized volatility, as \citeauthor{andersen2001} suggest, that under suitable conditions it provides an unbiased estimator of the return volatility. 


\begin{itemize}\itemsep0pt
\item S\&P 500 index data on daily basis
\item sampling period: 2000 - 2018
\item realized volatility: daily realized volatility of S\&P 500, calculated using 5 minute returns, retrieved from \citeauthor{heber2009}
\item model-free implied volatility: VIX index data
\item historic volatility: lagged realized volatility, for HAR-RV model use the average over the time period used to forecast
\end{itemize}


\subsection{Methodology: Linear Regression and HAR-RV model}
Consistent with prior research, for example \textcite{jiang2003}, \textcite{canina1993} or \textcite{christensen1998}, both univariate and encompassing regression analysis is used to analyse the information content of volatility measures. In the univariate regression, the realized volatility is only regressed on the historic data. For comparison, realized volatility is regressed on both historic data and the VIX in the encompassing regression analysis. Thus the encompassing regression analysis gives information about the relative importance of the volatility measures, and whether the VIX as one of them subsumes the information from the historic volatility. The two regressions are then given by:
\begin{align}
RV_{t+1d} = c + \beta^{d} RV^{d}_{t} + \beta^{w} RV^{w}_{t} + \beta^{m} RV^{m}_{t}  \\
RV_{t+1d} = c + \beta^{d} RV^{d}_{t} + \beta^{w} RV^{w}_{t} + \beta^{m} RV^{m}_{t} + \beta^{VIX} VIX
\end{align}
with .. (explain variables again?). In alignment with \textcite{corsi2009}, the Newey-West covariance correction is used, to account for the possible presence of serial correlation in the data, as serial correlation causes both the Gauss-Markow and Classical linear model assumptions to fail. 


\subsection{Limitations}
As volatility is stochastic, the ex-ante estimation will not equal the return volatility, as it is a measurement over an aggregated discrete time period \parencite{andersen2001}.
The VIX might be flawed, as \textcite{jiang2007} showed.

	\newpage
	%!TEX root = ../Main.tex

\section{Results}\label{sec:5Results}

This section presents the results, obtained with the regression analysis described in section \ref{sec:42Method}. The estimation results are summarized in \ref{tab:newey1} and \ref{tab:newey2}. First, concerning \textbf{H1}, if the VIX contains information about future volatility, the VIX slope coefficient should be different from zero. The results can not reject this hypothesis, in the univariate regression 

\subsection{Regression Analysis Results}\label{sec:51Regression}
%

% Table created by stargazer v.5.2.2 by Marek Hlavac, Harvard University. E-mail: hlavac at fas.harvard.edu
% Date and time: So, Jan 20, 2019 - 01:19:25
\begin{table}[!htbp] \centering 
\begin{threeparttable}
  \caption{Level regression (whole sample)} 
  \label{tab:newey1} 
\begin{tabular}{@{\extracolsep{5pt}}lccc} 
\\[-1.8ex]\hline 
\hline \\[-1.8ex] 
 & \multicolumn{3}{c}{\textit{Dependent variable:}} \\ 
\cline{2-4} 
\\[-1.8ex] & \multicolumn{3}{c}{Realized Volatility} \\ 
 & Reg1a & Reg2a & Reg3a \\ 
\\[-1.8ex] & (1) & (2) & (3)\\ 
\hline \\[-1.8ex] 
 Intercept & 0.045$^{***}$ & $-$0.324$^{***}$ & $-$0.169$^{***}$ \\ 
  & (0.015) & (0.059) & (0.034) \\ 
  & & & \\ 
 $RV^{(d)}_{t}$ & 0.362$^{***}$ &  & 0.256$^{***}$ \\ 
  & (0.038) &  & (0.040) \\ 
  & & & \\ 
 $RV^{(w)}_{t}$ & 0.391$^{***}$ &  & 0.286$^{***}$ \\ 
  & (0.056) &  & (0.064) \\ 
  & & & \\ 
 $RV^{(m)}_{t}$ & 0.188$^{***}$ &  & $-$0.106$^{**}$ \\ 
  & (0.036) &  & (0.050) \\ 
  & & & \\ 
 $crisis$ & 0.025$^{*}$ & $-$0.214$^{***}$ & $-$0.112$^{***}$ \\ 
  & (0.013) & (0.035) & (0.021) \\ 
  & & & \\ 
 $VIX_{t}$ &  & 1.052$^{***}$ & 0.579$^{***}$ \\ 
  &  & (0.059) & (0.064) \\ 
  & & & \\ 
\hline \\[-1.8ex] 
AIC & 2817.4 & 3104.2 & 2446 \\ 
Observations & 4,434 & 4,434 & 4,434 \\ 
R$^{2}$ & 0.708 & 0.689 & 0.732 \\ 
Adjusted R$^{2}$ & 0.708 & 0.688 & 0.732 \\ 
Residual Std. Error & 0.332 (df = 4429) & 0.343 (df = 4431) & 0.319 (df = 4428) \\ 
\hline 
\hline \\[-1.8ex] 
\textit{Note:}  & \multicolumn{3}{r}{$^{*}$p$<$0.1; $^{**}$p$<$0.05; $^{***}$p$<$0.01} \\ 
\end{tabular} 
\begin{tablenotes}
\small
\item  The numbers in the brackets are the standard errors of the parameters computed with Newey-West covariance correction, which are robust to autocorrelated and heteroscedastic error terms, see \textcite{newey1987}.
\end{tablenotes}
\end{threeparttable}
\end{table} 
\label{tab:newey1}
%

% Table created by stargazer v.5.2.2 by Marek Hlavac, Harvard University. E-mail: hlavac at fas.harvard.edu
% Date and time: So, Jan 20, 2019 - 01:19:26
\begin{table}[!htbp] \centering 
\begin{threeparttable}
  \caption{Logarithmic regression (whole sample)} 
  \label{tab:newey2} 
\begin{tabular}{@{\extracolsep{5pt}}lccc} 
\\[-1.8ex]\hline 
\hline \\[-1.8ex] 
 & \multicolumn{3}{c}{\textit{Dependent variable:}} \\ 
\cline{2-4} 
\\[-1.8ex] & \multicolumn{3}{c}{Realized Volatility} \\ 
 & Reg1b & Reg2b & Reg3b \\ 
\\[-1.8ex] & (1) & (2) & (3)\\ 
\hline \\[-1.8ex] 
 Intercept & $-$0.043$^{***}$ & $-$0.407$^{***}$ & $-$0.187$^{***}$ \\ 
  & (0.007) & (0.017) & (0.014) \\ 
  & & & \\ 
 $ln(RV^{(d)}_{t})$ & 0.344$^{***}$ &  & 0.263$^{***}$ \\ 
  & (0.027) &  & (0.025) \\ 
  & & & \\ 
 $ln(RV^{(w)}_{t})$ & 0.395$^{***}$ &  & 0.284$^{***}$ \\ 
  & (0.035) &  & (0.035) \\ 
  & & & \\ 
 $ ln(RV^{(m)}_{t})$ & 0.208$^{***}$ &  & 0.014 \\ 
  & (0.024) &  & (0.029) \\ 
  & & & \\ 
 $crisis$ & 0.020$^{*}$ & $-$0.224$^{***}$ & $-$0.100$^{***}$ \\ 
  & (0.012) & (0.034) & (0.016) \\ 
  & & & \\ 
 $ln(VIX_{t})$ &  & 1.472$^{***}$ & 0.648$^{***}$ \\ 
  &  & (0.048) & (0.045) \\ 
  & & & \\ 
\hline \\[-1.8ex] 
AIC & 1874.2 & 2363.6 & 1551.6 \\ 
Observations & 4,434 & 4,434 & 4,434 \\ 
R$^{2}$ & 0.726 & 0.694 & 0.745 \\ 
Adjusted R$^{2}$ & 0.726 & 0.693 & 0.745 \\ 
Residual Std. Error & 0.299 (df = 4429) & 0.316 (df = 4431) & 0.288 (df = 4428) \\ 
\hline 
\hline \\[-1.8ex] 
\textit{Note:}  & \multicolumn{3}{r}{$^{*}$p$<$0.1; $^{**}$p$<$0.05; $^{***}$p$<$0.01} \\ 
\end{tabular} 
\begin{tablenotes}
\small
\item  The numbers in the brackets are the standard errors of the parameters computed with Newey-West covariance correction, which are robust to autocorrelated and heteroscedastic error terms, see \textcite{newey1987}.
\end{tablenotes}
\end{threeparttable}
\end{table} 
\label{tab:newey2}
%
%
% Table created by stargazer v.5.2.2 by Marek Hlavac, Harvard University. E-mail: hlavac at fas.harvard.edu
% Date and time: Fr, Jan 11, 2019 - 16:29:43
\begin{table}[!htbp] \centering 
  \caption{level regression} 
  \label{} 
\begin{tabular}{@{\extracolsep{5pt}}lccc} 
\\[-1.8ex]\hline 
\hline \\[-1.8ex] 
 & \multicolumn{3}{c}{\textit{Dependent variable:}} \\ 
\cline{2-4} 
\\[-1.8ex] & \multicolumn{3}{c}{Realized Volatility} \\ 
\\[-1.8ex] & (1) & (2) & (3)\\ 
\hline \\[-1.8ex] 
 Intercept & 0.045$^{***}$ & $-$0.238$^{***}$ & $-$0.015 \\ 
  & (0.015) & (0.032) & (0.021) \\ 
  & & & \\ 
 $RV^{d}$ & 0.362$^{***}$ &  & 0.111$^{**}$ \\ 
  & (0.038) &  & (0.049) \\ 
  & & & \\ 
 $RV^{w}$ & 0.391$^{***}$ &  & 0.277$^{***}$ \\ 
  & (0.056) &  & (0.063) \\ 
  & & & \\ 
 $RV^{m}$ & 0.188$^{***}$ &  & 0.364$^{***}$ \\ 
  & (0.036) &  & (0.072) \\ 
  & & & \\ 
 crisis & 0.025$^{*}$ & $-$0.164$^{***}$ & $-$0.015 \\ 
  & (0.013) & (0.031) & (0.018) \\ 
  & & & \\ 
 VIX &  & 1.562$^{***}$ & 1.388$^{***}$ \\ 
  &  & (0.110) & (0.125) \\ 
  & & & \\ 
 weekVIX &  & 0.005 & $-$0.694$^{***}$ \\ 
  &  & (0.157) & (0.141) \\ 
  & & & \\ 
 monthVIX &  & $-$0.599$^{***}$ & $-$0.497$^{***}$ \\ 
  &  & (0.131) & (0.114) \\ 
  & & & \\ 
\hline \\[-1.8ex] 
AIC & 2817.4 & 2673.3 & 2033.4 \\ 
Observations & 4,434 & 4,434 & 4,434 \\ 
R$^{2}$ & 0.708 & 0.718 & 0.756 \\ 
Adjusted R$^{2}$ & 0.708 & 0.717 & 0.756 \\ 
Residual Std. Error & 0.332 (df = 4429) & 0.327 (df = 4429) & 0.304 (df = 4426) \\ 
\hline 
\hline \\[-1.8ex] 
\textit{Note:}  & \multicolumn{3}{r}{$^{*}$p$<$0.1; $^{**}$p$<$0.05; $^{***}$p$<$0.01} \\ 
\end{tabular} 
\end{table} 
\label{tab:newey3}
%
%
% Table created by stargazer v.5.2.2 by Marek Hlavac, Harvard University. E-mail: hlavac at fas.harvard.edu
% Date and time: Sa, Jan 12, 2019 - 12:14:27
\begin{table}[!htbp] \centering 
  \caption{logarithmic regression} 
  \label{} 
\begin{tabular}{@{\extracolsep{5pt}}lccc} 
\\[-1.8ex]\hline 
\hline \\[-1.8ex] 
 & \multicolumn{3}{c}{\textit{Dependent variable:}} \\ 
\cline{2-4} 
\\[-1.8ex] & \multicolumn{3}{c}{Realized Volatility} \\ 
\\[-1.8ex] & (1) & (2) & (3)\\ 
\hline \\[-1.8ex] 
 Intercept & $-$0.043$^{***}$ & $-$0.404$^{***}$ & $-$0.096$^{***}$ \\ 
  & (0.007) & (0.016) & (0.011) \\ 
  & & & \\ 
 $RV^{d}_{log}$ & 0.344$^{***}$ &  & 0.126$^{***}$ \\ 
  & (0.027) &  & (0.024) \\ 
  & & & \\ 
 $RV^{w}_{log}$ & 0.395$^{***}$ &  & 0.341$^{***}$ \\ 
  & (0.035) &  & (0.039) \\ 
  & & & \\ 
 $RV^{m}_{log}$ & 0.208$^{***}$ &  & 0.351$^{***}$ \\ 
  & (0.024) &  & (0.040) \\ 
  & & & \\ 
 crisis & 0.020$^{*}$ & $-$0.195$^{***}$ & $-$0.021 \\ 
  & (0.012) & (0.033) & (0.013) \\ 
  & & & \\ 
 $VIX1{d}_{log}$ &  & 2.039$^{***}$ & 1.781$^{***}$ \\ 
  &  & (0.086) & (0.082) \\ 
  & & & \\ 
 $VIX^{w}_{log}$ &  & $-$0.222$^{*}$ & $-$1.098$^{***}$ \\ 
  &  & (0.121) & (0.118) \\ 
  & & & \\ 
 $VIX^{m}_{log}$ &  & $-$0.412$^{***}$ & $-$0.468$^{***}$ \\ 
  &  & (0.104) & (0.078) \\ 
  & & & \\ 
\hline \\[-1.8ex] 
AIC & 1874.2 & 2215.4 & 1174.6 \\ 
Observations & 4,434 & 4,434 & 4,434 \\ 
R$^{2}$ & 0.726 & 0.704 & 0.766 \\ 
Adjusted R$^{2}$ & 0.726 & 0.704 & 0.766 \\ 
Residual Std. Error & 0.299 (df = 4429) & 0.310 (df = 4429) & 0.276 (df = 4426) \\ 
\hline 
\hline \\[-1.8ex] 
\textit{Note:}  & \multicolumn{3}{r}{$^{*}$p$<$0.1; $^{**}$p$<$0.05; $^{***}$p$<$0.01} \\ 
\end{tabular} 
\end{table} 
\label{tab:newey3}

\subsection{Robustness Checks}\label{sec51Robustness}


\subsubsection{Monthly non-overlapping samples}
Previous samples testing the information content of (model-free) implied volatility often used overlapping samples, meaning that the same option is used in several implied-volatility calculations. However, \textcite{christensen1998} showed, that the use of overlapping samples creates a telescopic overlap problem and thus standard statistical inferences are no longer valid.\\
Therefore the same regression analysis was conducted using non-overlapping samples. \textcite{jiang2003} use monthly non-overlapping samples, using the first Wednesday of every month, since they calculate the implied volatility over a horizon on one month. The VIX however is calculated slightly differently. It contains near- and next-term options options between 23 and 37 days to maturity (which is always a Friday), and every week the options roll over to new maturities. For example, taking the second Tuesday in October, the near-term option expires in 24 days, and the next-term option in 31 days. One day later, the option that expires now in 30 days is the near-term option, and another option expiring in 37 days is the next-term option. This next-term option will, one week later, roll over to a near-term option and, one more week later, drop out of the calculation. Thus, an option can be included in the calculation for up to two weeks. Therefore, the regression is conducted with daily volatilities, but only for one value out of two weeks. As in \textcite{jiang2003}, the values of Wednesday are used, for each second week. \\
The estimation results for the sample using non-overlapping data are summarized in \ref{tab:overlap1} and \ref{tab:overlap2}.


% Table created by stargazer v.5.2.2 by Marek Hlavac, Harvard University. E-mail: hlavac at fas.harvard.edu
% Date and time: Mi, Jan 16, 2019 - 19:14:31
\begin{table}[!htbp] \centering 
  \caption{Level regression (non-overlapping sample)} 
  \label{tab:overlap1} 
\begin{tabular}{@{\extracolsep{5pt}}lccc} 
\\[-1.8ex]\hline 
\hline \\[-1.8ex] 
 & \multicolumn{3}{c}{\textit{Dependent variable:}} \\ 
\cline{2-4} 
\\[-1.8ex] & \multicolumn{3}{c}{Realized Volatility} \\ 
 & Reg1a & Reg2a & Reg3a \\ 
\\[-1.8ex] & (1) & (2) & (3)\\ 
\hline \\[-1.8ex] 
 Intercept & 0.046 & $-$0.342$^{***}$ & $-$0.134$^{**}$ \\ 
  & (0.031) & (0.091) & (0.052) \\ 
  & & & \\ 
 $RV^{(d)}_{t}$ & 0.408$^{***}$ &  & 0.330$^{***}$ \\ 
  & (0.109) &  & (0.116) \\ 
  & & & \\ 
 $RV^{(w)}_{t}$ & 0.501$^{***}$ &  & 0.394$^{***}$ \\ 
  & (0.126) &  & (0.128) \\ 
  & & & \\ 
 $RV^{(m)}_{t}$ & 0.072 &  & $-$0.132 \\ 
  & (0.079) &  & (0.083) \\ 
  & & & \\ 
 $crisis$ & $-$0.022 & $-$0.257$^{***}$ & $-$0.131$^{***}$ \\ 
  & (0.029) & (0.058) & (0.035) \\ 
  & & & \\ 
 $VIX_{t}$ &  & 1.092$^{***}$ & 0.460$^{***}$ \\ 
  &  & (0.094) & (0.089) \\ 
  & & & \\ 
\hline \\[-1.8ex] 
AIC & 192.5 & 281.8 & 166 \\ 
Observations & 456 & 456 & 456 \\ 
R$^{2}$ & 0.757 & 0.701 & 0.771 \\ 
Adjusted R$^{2}$ & 0.754 & 0.700 & 0.769 \\ 
Residual Std. Error & 0.297 (df = 451) & 0.328 (df = 453) & 0.288 (df = 450) \\ 
\hline 
\hline \\[-1.8ex] 
\textit{Note:}  & \multicolumn{3}{r}{$^{*}$p$<$0.1; $^{**}$p$<$0.05; $^{***}$p$<$0.01} \\ 
\end{tabular} 
\end{table} 
\label{tab:overlap1}
%

% Table created by stargazer v.5.2.2 by Marek Hlavac, Harvard University. E-mail: hlavac at fas.harvard.edu
% Date and time: So, Jan 13, 2019 - 10:25:28
\begin{table}[!htbp] \centering 
  \caption{logarithmic regression} 
  \label{} 
\begin{tabular}{@{\extracolsep{5pt}}lccc} 
\\[-1.8ex]\hline 
\hline \\[-1.8ex] 
 & \multicolumn{3}{c}{\textit{Dependent variable:}} \\ 
\cline{2-4} 
\\[-1.8ex] & \multicolumn{3}{c}{Realized Volatility} \\ 
 & Reg1b & Reg2b & Reg3b \\ 
\\[-1.8ex] & (1) & (2) & (3)\\ 
\hline \\[-1.8ex] 
 c & $-$0.001 & $-$0.364$^{***}$ & $-$0.049$^{*}$ \\ 
  & (0.018) & (0.026) & (0.028) \\ 
  & & & \\ 
 $ ln(RV^{d}_{t})$ & 0.346$^{***}$ &  & 0.352$^{***}$ \\ 
  & (0.064) &  & (0.064) \\ 
  & & & \\ 
 $ln(RV^{w}_{t})$ & 0.408$^{***}$ &  & 0.432$^{***}$ \\ 
  & (0.078) &  & (0.080) \\ 
  & & & \\ 
 $ ln(RV^{m}_{t})$ & 0.171$^{***}$ &  & 0.003 \\ 
  & (0.054) &  & (0.088) \\ 
  & & & \\ 
 $crisis$ & $-$0.017 & $-$0.169$^{***}$ & $-$0.063$^{*}$ \\ 
  & (0.027) & (0.062) & (0.034) \\ 
  & & & \\ 
 $ln(VIX_{t})$ &  & 1.229$^{***}$ & 0.236$^{**}$ \\ 
  &  & (0.074) & (0.104) \\ 
  & & & \\ 
\hline \\[-1.8ex] 
AIC & 126.4 & 415.1 & 123.9 \\ 
Observations & 456 & 455 & 455 \\ 
R$^{2}$ & 0.748 & 0.522 & 0.751 \\ 
Adjusted R$^{2}$ & 0.746 & 0.519 & 0.748 \\ 
Residual Std. Error & 0.276 (df = 451) & 0.380 (df = 452) & 0.275 (df = 449) \\ 
\hline 
\hline \\[-1.8ex] 
\textit{Note:}  & \multicolumn{3}{r}{$^{*}$p$<$0.1; $^{**}$p$<$0.05; $^{***}$p$<$0.01} \\ 
\end{tabular} 
\end{table} 
\label{tab:overlap2}
%
%
% Table created by stargazer v.5.2.2 by Marek Hlavac, Harvard University. E-mail: hlavac at fas.harvard.edu
% Date and time: Sa, Jan 12, 2019 - 12:17:26
\begin{table}[!htbp] \centering 
  \caption{level regression} 
  \label{} 
\begin{tabular}{@{\extracolsep{5pt}}lccc} 
\\[-1.8ex]\hline 
\hline \\[-1.8ex] 
 & \multicolumn{3}{c}{\textit{Dependent variable:}} \\ 
\cline{2-4} 
\\[-1.8ex] & \multicolumn{3}{c}{Realized Volatility} \\ 
\\[-1.8ex] & (1) & (2) & (3)\\ 
\hline \\[-1.8ex] 
 Intercept & 0.046 & $-$0.270$^{***}$ & $-$0.037 \\ 
  & (0.031) & (0.051) & (0.039) \\ 
  & & & \\ 
 $RV^{d}$ & 0.408$^{***}$ &  & 0.274$^{**}$ \\ 
  & (0.109) &  & (0.117) \\ 
  & & & \\ 
 $RV^{w}$ & 0.501$^{***}$ &  & 0.142 \\ 
  & (0.126) &  & (0.228) \\ 
  & & & \\ 
 $RV^{m}$ & 0.072 &  & 0.357 \\ 
  & (0.079) &  & (0.222) \\ 
  & & & \\ 
 crisis & $-$0.022 & $-$0.210$^{***}$ & $-$0.065$^{***}$ \\ 
  & (0.029) & (0.050) & (0.025) \\ 
  & & & \\ 
 $VIX^{d}$ &  & 1.328$^{***}$ & 0.856$^{***}$ \\ 
  &  & (0.238) & (0.170) \\ 
  & & & \\ 
 $VIX^{w}$ &  & 0.474$^{*}$ & 0.124 \\ 
  &  & (0.258) & (0.330) \\ 
  & & & \\ 
 $VIX^{m}$ &  & $-$0.779$^{***}$ & $-$0.749$^{**}$ \\ 
  &  & (0.174) & (0.315) \\ 
  & & & \\ 
\hline \\[-1.8ex] 
AIC & 192.5 & 229 & 137.8 \\ 
Observations & 456 & 456 & 456 \\ 
R$^{2}$ & 0.757 & 0.736 & 0.787 \\ 
Adjusted R$^{2}$ & 0.754 & 0.734 & 0.784 \\ 
Residual Std. Error & 0.297 (df = 451) & 0.309 (df = 451) & 0.278 (df = 448) \\ 
\hline 
\hline \\[-1.8ex] 
\textit{Note:}  & \multicolumn{3}{r}{$^{*}$p$<$0.1; $^{**}$p$<$0.05; $^{***}$p$<$0.01} \\ 
\end{tabular} 
\end{table} 
\label{tab:overlap3}

%
% Table created by stargazer v.5.2.2 by Marek Hlavac, Harvard University. E-mail: hlavac at fas.harvard.edu
% Date and time: Sa, Jan 12, 2019 - 12:17:28
\begin{table}[!htbp] \centering 
  \caption{logarithmic regression} 
  \label{} 
\begin{tabular}{@{\extracolsep{5pt}}lccc} 
\\[-1.8ex]\hline 
\hline \\[-1.8ex] 
 & \multicolumn{3}{c}{\textit{Dependent variable:}} \\ 
\cline{2-4} 
\\[-1.8ex] & \multicolumn{3}{c}{Realized Volatility} \\ 
\\[-1.8ex] & (1) & (2) & (3)\\ 
\hline \\[-1.8ex] 
 Intercept & $-$0.001 & $-$0.364$^{***}$ & $-$0.098$^{***}$ \\ 
  & (0.018) & (0.022) & (0.031) \\ 
  & & & \\ 
 $RV^{d}_{log}$ & 0.346$^{***}$ &  & 0.172$^{***}$ \\ 
  & (0.064) &  & (0.061) \\ 
  & & & \\ 
 $RV^{w}_{log}$ & 0.408$^{***}$ &  & 0.267$^{**}$ \\ 
  & (0.078) &  & (0.103) \\ 
  & & & \\ 
 $RV^{m}_{log}$ & 0.171$^{***}$ &  & 0.252$^{**}$ \\ 
  & (0.054) &  & (0.102) \\ 
  & & & \\ 
 crisis & $-$0.017 & $-$0.241$^{***}$ & $-$0.094$^{***}$ \\ 
  & (0.027) & (0.045) & (0.035) \\ 
  & & & \\ 
 $VIX1{d}_{log}$ &  & 1.743$^{***}$ & 1.350$^{***}$ \\ 
  &  & (0.195) & (0.185) \\ 
  & & & \\ 
 $VIX^{w}_{log}$ &  & 0.157 & $-$0.435 \\ 
  &  & (0.253) & (0.326) \\ 
  & & & \\ 
 $VIX^{m}_{log}$ &  & $-$0.479$^{***}$ & $-$0.503$^{**}$ \\ 
  &  & (0.155) & (0.206) \\ 
  & & & \\ 
\hline \\[-1.8ex] 
AIC & 126.4 & 163.6 & 66.1 \\ 
Observations & 456 & 456 & 456 \\ 
R$^{2}$ & 0.748 & 0.727 & 0.782 \\ 
Adjusted R$^{2}$ & 0.746 & 0.724 & 0.779 \\ 
Residual Std. Error & 0.276 (df = 451) & 0.287 (df = 451) & 0.257 (df = 448) \\ 
\hline 
\hline \\[-1.8ex] 
\textit{Note:}  & \multicolumn{3}{r}{$^{*}$p$<$0.1; $^{**}$p$<$0.05; $^{***}$p$<$0.01} \\ 
\end{tabular} 
\end{table} 
\label{tab:overlap4}

\subsubsection{IV Regression}

















	\newpage
	%!TEX root = ../Main.tex

\section{Discussion}


	\newpage
	%!TEX root = ../Main.tex
\newgeometry{left=1cm, right=1cm, top=2cm, bottom=1cm}
\section*{Appendix}
\appendix
\section*{Figures}
%
\begin{figure}[!htbp]\caption{S\&P 500 and VIX}\label{fig:SPandVIX}
\centering
\includegraphics[width=18cm, height=8cm]{pictures/SPandViX.png}
\end{figure}
%
\begin{figure}[!htbp]\caption{S\&P 500, RV and VIX}\label{fig:SPandVIXandVol}
\centering
\includegraphics[width=18cm, height=8cm]{pictures/SPandVolandViX.png}
\end{figure}
\restoregeometry

\newpage
\section*{Data Summary Statistics}
% Table created by stargazer v.5.2.2 by Marek Hlavac, Harvard University. E-mail: hlavac at fas.harvard.edu
% Date and time: Fr, Jan 11, 2019 - 17:33:58
\begin{table}[!htbp] \centering 
  \caption{Summary statistics: Variables} 
  \label{tab:summary1} 
\begin{tabular}{@{\extracolsep{5pt}}lccccccc} 
\\[-1.8ex]\hline 
\hline \\[-1.8ex] 
Statistic & \multicolumn{1}{c}{N} & \multicolumn{1}{c}{Min} & \multicolumn{1}{c}{Max} & \multicolumn{1}{c}{Mean} & \multicolumn{1}{c}{St. Dev.} & \multicolumn{1}{c}{Skewness} & \multicolumn{1}{c}{Kurtosis} \\
\hline \\[-1.8ex] 
\multicolumn{8}{c}{All time periods} \\
\hline \\[-1.8ex] 
RV & 4,434 & 0.110 & 8.802 & 0.874 & 0.615 & 3.127 & 17.751 \\ 
VIX & 4,434 & 0.576 & 5.094 & 1.196 & 0.525 & 2.552 & 9.787 \\ 
Weekly RV & 4,434 & 0.183 & 5.586 & 0.874 & 0.556 & 2.757 & 12.451 \\ 
Monthly RV & 4,434 & 0.223 & 4.375 & 0.876 & 0.512 & 2.534 & 10.051\\ 
Weekly VIX & 4,434 & 0.596 & 4.593 & 1.197 & 0.519 & 2.510 & 9.235 \\ 
Monthly VIX & 4,434 & 0.618 & 4.126 & 1.198 & 0.504 & 2.491 & 8.853 \\ 
\hline \\[-1.8ex] 
\multicolumn{8}{c}{During crisis} \\
\hline \\[-1.8ex] 
RV & 1,252 & 0.213 & 8.802 & 1.171 & 0.834 & 2.713 & 11.399\\ 
VIX  & 1,259 & 0.847 & 5.094 & 1.623 & 0.695  & 1.941 &  4.232\\ 
Weekly RV & 1,259 & 0.257 & 5.586 & 1.169 & 0.753 & 2.409 & 7.417\\ 
Monthly RV & 1,259 & 0.423 & 4.375 & 1.172 & 0.689 & 2.178 & 5.448\\ 
Weekly VIX & 1,259 & 0.885 & 4.593 & 1.623 & 0.684  &  1.885 & 3.753\\ 
Monthly VIX & 1,259 & 0.946 & 4.126 & 1.625 & 0.662 & 1.837 & 3.311\\ 
\hline \\[-1.8ex] 
\multicolumn{8}{c}{Outside of crisis} \\
\hline \\[-1.8ex] 
RV & 3,229 & 0.110 & 6.109 & 0.762 & 0.454 & 2.252 & 10.798\\ 
VIX & 3,251 & 0.576 & 2.566 & 1.034 & 0.307 & 1.206 & 1.365 \\ 
Weekly RV & 3,246 & 0.183 & 3.165 & 0.765 & 0.400 & 1.564 &  3.441\\ 
Monthly RV & 3,229 & 0.223 & 2.367 & 0.764 & 0.363 & 1.308 & 1.687  \\ 
Weekly VIX & 3,246 & 0.596 & 2.188 & 1.034 & 0.300  & 1.157 &  1.047 \\ 
Monthly VIX & 3,229 & 0.618 & 2.014 & 1.033 & 0.285 &  1.1 & 0.78\\ 
\hline \\[-1.8ex] 
\end{tabular} 
\end{table} 

\newgeometry{left=1cm, right=1cm, top=2cm, bottom=1cm}

% Table created by stargazer v.5.2.2 by Marek Hlavac, Harvard University. E-mail: hlavac at fas.harvard.edu
% Date and time: Fr, Jan 11, 2019 - 18:36:21
\begin{table}[!htbp] \centering 
  \caption{Summary statistics: Logarithm variables} 
  \label{tab:summary2} 
\begin{tabular}{@{\extracolsep{5pt}}lccccccc} 
\\[-1.8ex]\hline 
\hline \\[-1.8ex] 
Statistic & \multicolumn{1}{c}{N} & \multicolumn{1}{c}{Min} & \multicolumn{1}{c}{Max} & \multicolumn{1}{c}{Mean} & \multicolumn{1}{c}{St. Dev.} & \multicolumn{1}{c}{Skewness} & \multicolumn{1}{c}{Kurtosis} \\ 
\hline \\[-1.8ex] 
\multicolumn{8}{c}{All time periods} \\
\hline \\[-1.8ex] 
RV & 1,236 & $-$9.664 & 0.777 & $-$1.469 & 1.262 & -1.369 & 3.185 \\ 
VIX & 2,528 & $-$8.005 & 0.487 & $-$1.562 & 1.190 & -1.294 & 2.586\\ 
Weekly RV & 4,434 & 0.183 & 5.586 & 0.874 & 0.556  &  2.757 & 12.451\\ 
Monthly RV & 4,434 & $-$1.501 & 1.476 & $-$0.259 & 0.483 &  0.487 & 0.357  \\ 
Weekly VIX & 4,434 & $-$0.518 & 1.525 & 0.111 & 0.350 & 0.933 &  1.117  \\ 
Monthly VIX & 4,434 & $-$0.481 & 1.417 & 0.115 & 0.340 &  0.963 &   1.227\\ 
\hline \\[-1.8ex] 
\multicolumn{8}{c}{During crisis} \\
\hline \\[-1.8ex] 
RV & 1,252 & 0.213 & 8.802 & 1.171 & 0.834 & 2.713 & 11.399\\ 
VIX & 1,259 & 0.847 & 5.094 & 1.623 & 0.695 & 1.941 & 4.232 \\ 
Weekly RV & 1,259 & 0.257 & 5.586 & 1.169 & 0.753 &  2.409 & 7.417 \\ 
Monthly RV & 1,259 & 0.423 & 4.375 & 1.172 & 0.689 & 2.178 & 5.448\\ 
Weekly VIX & 1,259 & 0.885 & 4.593 & 1.623 & 0.684 &  1.885 &  3.753\\ 
Monthly VIX & 1,259 & 0.946 & 4.126 & 1.625 & 0.662 &  1.837 &  3.311\\ 
\hline \\[-1.8ex] 
\multicolumn{8}{c}{Outside of crisis} \\                           
\hline \\[-1.8ex] 
RV & 3,229 & 0.110 & 6.109 & 0.762 & 0.454 & 2.252  & 10.798\\ 
VIX & 3,251 & 0.576 & 2.566 & 1.034 & 0.307 & 1.206 & 1.365\\ 
Weekly RV & 3,246 & 0.183 & 3.165 & 0.765 & 0.400  &  1.564  & 3.441 \\ 
Monthly RV & 3,229 & 0.223 & 2.367 & 0.764 & 0.363  & 1.308 & 1.687\\ 
Weekly VIX & 3,246 & 0.596 & 2.188 & 1.034 & 0.300  & 1.157 & 1.047\\ 
Monthly VIX & 3,229 & 0.618 & 2.014 & 1.033 & 0.285 & 1.100 & 0.780 \\ 
\hline \\[-1.8ex] 
\end{tabular} 
\end{table} 

% Table created by stargazer v.5.2.2 by Marek Hlavac, Harvard University. E-mail: hlavac at fas.harvard.edu
% Date and time: Sa, Jan 12, 2019 - 09:20:14
\begin{table}[!htbp] \centering 
  \caption{Correlation table} 
  \label{tab:correlation} 
\begin{tabular}{@{\extracolsep{5pt}} ccccccccc} 
\\[-1.8ex]\hline 
\hline \\[-1.8ex] 
 & RV & VIX & Daily RV & Weekly RV & Monthly RV & D. VIX & W.VIX & M. VIX \\ 
\hline \\[-1.8ex] 
RV & $1$ & $0.837$ & $0.806$ & $0.823$ & $0.769$ & $0.819$ & $0.781$ & $0.698$ \\ 
VIX & $0.837$ & $1$ & $0.822$ & $0.883$ & $0.903$ & $0.981$ & $0.969$ & $0.926$ \\ 
Daily RV & $0.806$ & $0.822$ & $1$ & $0.890$ & $0.794$ & $0.837$ & $0.803$ & $0.710$ \\ 
Weekly RV & $0.823$ & $0.883$ & $0.890$ & $1$ & $0.903$ & $0.896$ & $0.906$ & $0.809$ \\ 
Monthly RV & $0.769$ & $0.903$ & $0.794$ & $0.903$ & $1$ & $0.910$ & $0.932$ & $0.931$ \\ 
Daily VIX & $0.819$ & $0.981$ & $0.837$ & $0.896$ & $0.910$ & $1$ & $0.982$ & $0.935$ \\ 
Weekly VIX & $0.781$ & $0.969$ & $0.803$ & $0.906$ & $0.932$ & $0.982$ & $1$ & $0.961$ \\ 
Monthly VIX & $0.698$ & $0.926$ & $0.710$ & $0.809$ & $0.931$ & $0.935$ & $0.961$ & $1$ \\ 
\hline \\[-1.8ex] 
\end{tabular} 
\end{table} 

\restoregeometry
%\newgeometry{left=2cm, right=2cm, top=2cm, bottom=1cm}
\section*{Regression Results with Robustness Checks}

% Table created by stargazer v.5.2.2 by Marek Hlavac, Harvard University. E-mail: hlavac at fas.harvard.edu
% Date and time: So, Jan 20, 2019 - 01:19:25
\begin{table}[!htbp] \centering 
\begin{threeparttable}
  \caption{Level regression (whole sample)} 
  \label{tab:newey1} 
\begin{tabular}{@{\extracolsep{5pt}}lccc} 
\\[-1.8ex]\hline 
\hline \\[-1.8ex] 
 & \multicolumn{3}{c}{\textit{Dependent variable:}} \\ 
\cline{2-4} 
\\[-1.8ex] & \multicolumn{3}{c}{Realized Volatility} \\ 
 & Reg1a & Reg2a & Reg3a \\ 
\\[-1.8ex] & (1) & (2) & (3)\\ 
\hline \\[-1.8ex] 
 Intercept & 0.045$^{***}$ & $-$0.324$^{***}$ & $-$0.169$^{***}$ \\ 
  & (0.015) & (0.059) & (0.034) \\ 
  & & & \\ 
 $RV^{(d)}_{t}$ & 0.362$^{***}$ &  & 0.256$^{***}$ \\ 
  & (0.038) &  & (0.040) \\ 
  & & & \\ 
 $RV^{(w)}_{t}$ & 0.391$^{***}$ &  & 0.286$^{***}$ \\ 
  & (0.056) &  & (0.064) \\ 
  & & & \\ 
 $RV^{(m)}_{t}$ & 0.188$^{***}$ &  & $-$0.106$^{**}$ \\ 
  & (0.036) &  & (0.050) \\ 
  & & & \\ 
 $crisis$ & 0.025$^{*}$ & $-$0.214$^{***}$ & $-$0.112$^{***}$ \\ 
  & (0.013) & (0.035) & (0.021) \\ 
  & & & \\ 
 $VIX_{t}$ &  & 1.052$^{***}$ & 0.579$^{***}$ \\ 
  &  & (0.059) & (0.064) \\ 
  & & & \\ 
\hline \\[-1.8ex] 
AIC & 2817.4 & 3104.2 & 2446 \\ 
Observations & 4,434 & 4,434 & 4,434 \\ 
R$^{2}$ & 0.708 & 0.689 & 0.732 \\ 
Adjusted R$^{2}$ & 0.708 & 0.688 & 0.732 \\ 
Residual Std. Error & 0.332 (df = 4429) & 0.343 (df = 4431) & 0.319 (df = 4428) \\ 
\hline 
\hline \\[-1.8ex] 
\textit{Note:}  & \multicolumn{3}{r}{$^{*}$p$<$0.1; $^{**}$p$<$0.05; $^{***}$p$<$0.01} \\ 
\end{tabular} 
\begin{tablenotes}
\small
\item  The numbers in the brackets are the standard errors of the parameters computed with Newey-West covariance correction, which are robust to autocorrelated and heteroscedastic error terms, see \textcite{newey1987}.
\end{tablenotes}
\end{threeparttable}
\end{table} 


% Table created by stargazer v.5.2.2 by Marek Hlavac, Harvard University. E-mail: hlavac at fas.harvard.edu
% Date and time: Mi, Jan 16, 2019 - 19:14:31
\begin{table}[!htbp] \centering 
  \caption{Level regression (non-overlapping sample)} 
  \label{tab:overlap1} 
\begin{tabular}{@{\extracolsep{5pt}}lccc} 
\\[-1.8ex]\hline 
\hline \\[-1.8ex] 
 & \multicolumn{3}{c}{\textit{Dependent variable:}} \\ 
\cline{2-4} 
\\[-1.8ex] & \multicolumn{3}{c}{Realized Volatility} \\ 
 & Reg1a & Reg2a & Reg3a \\ 
\\[-1.8ex] & (1) & (2) & (3)\\ 
\hline \\[-1.8ex] 
 Intercept & 0.046 & $-$0.342$^{***}$ & $-$0.134$^{**}$ \\ 
  & (0.031) & (0.091) & (0.052) \\ 
  & & & \\ 
 $RV^{(d)}_{t}$ & 0.408$^{***}$ &  & 0.330$^{***}$ \\ 
  & (0.109) &  & (0.116) \\ 
  & & & \\ 
 $RV^{(w)}_{t}$ & 0.501$^{***}$ &  & 0.394$^{***}$ \\ 
  & (0.126) &  & (0.128) \\ 
  & & & \\ 
 $RV^{(m)}_{t}$ & 0.072 &  & $-$0.132 \\ 
  & (0.079) &  & (0.083) \\ 
  & & & \\ 
 $crisis$ & $-$0.022 & $-$0.257$^{***}$ & $-$0.131$^{***}$ \\ 
  & (0.029) & (0.058) & (0.035) \\ 
  & & & \\ 
 $VIX_{t}$ &  & 1.092$^{***}$ & 0.460$^{***}$ \\ 
  &  & (0.094) & (0.089) \\ 
  & & & \\ 
\hline \\[-1.8ex] 
AIC & 192.5 & 281.8 & 166 \\ 
Observations & 456 & 456 & 456 \\ 
R$^{2}$ & 0.757 & 0.701 & 0.771 \\ 
Adjusted R$^{2}$ & 0.754 & 0.700 & 0.769 \\ 
Residual Std. Error & 0.297 (df = 451) & 0.328 (df = 453) & 0.288 (df = 450) \\ 
\hline 
\hline \\[-1.8ex] 
\textit{Note:}  & \multicolumn{3}{r}{$^{*}$p$<$0.1; $^{**}$p$<$0.05; $^{***}$p$<$0.01} \\ 
\end{tabular} 
\end{table} 


% Table created by stargazer v.5.2.2 by Marek Hlavac, Harvard University. E-mail: hlavac at fas.harvard.edu
% Date and time: So, Jan 13, 2019 - 10:25:28
\begin{table}[!htbp] \centering 
  \caption{logarithmic regression} 
  \label{} 
\begin{tabular}{@{\extracolsep{5pt}}lccc} 
\\[-1.8ex]\hline 
\hline \\[-1.8ex] 
 & \multicolumn{3}{c}{\textit{Dependent variable:}} \\ 
\cline{2-4} 
\\[-1.8ex] & \multicolumn{3}{c}{Realized Volatility} \\ 
 & Reg1b & Reg2b & Reg3b \\ 
\\[-1.8ex] & (1) & (2) & (3)\\ 
\hline \\[-1.8ex] 
 c & $-$0.001 & $-$0.364$^{***}$ & $-$0.049$^{*}$ \\ 
  & (0.018) & (0.026) & (0.028) \\ 
  & & & \\ 
 $ ln(RV^{d}_{t})$ & 0.346$^{***}$ &  & 0.352$^{***}$ \\ 
  & (0.064) &  & (0.064) \\ 
  & & & \\ 
 $ln(RV^{w}_{t})$ & 0.408$^{***}$ &  & 0.432$^{***}$ \\ 
  & (0.078) &  & (0.080) \\ 
  & & & \\ 
 $ ln(RV^{m}_{t})$ & 0.171$^{***}$ &  & 0.003 \\ 
  & (0.054) &  & (0.088) \\ 
  & & & \\ 
 $crisis$ & $-$0.017 & $-$0.169$^{***}$ & $-$0.063$^{*}$ \\ 
  & (0.027) & (0.062) & (0.034) \\ 
  & & & \\ 
 $ln(VIX_{t})$ &  & 1.229$^{***}$ & 0.236$^{**}$ \\ 
  &  & (0.074) & (0.104) \\ 
  & & & \\ 
\hline \\[-1.8ex] 
AIC & 126.4 & 415.1 & 123.9 \\ 
Observations & 456 & 455 & 455 \\ 
R$^{2}$ & 0.748 & 0.522 & 0.751 \\ 
Adjusted R$^{2}$ & 0.746 & 0.519 & 0.748 \\ 
Residual Std. Error & 0.276 (df = 451) & 0.380 (df = 452) & 0.275 (df = 449) \\ 
\hline 
\hline \\[-1.8ex] 
\textit{Note:}  & \multicolumn{3}{r}{$^{*}$p$<$0.1; $^{**}$p$<$0.05; $^{***}$p$<$0.01} \\ 
\end{tabular} 
\end{table} 


\section*{F-test Results with Robustness Checks}
% latex table generated in R 3.5.1 by xtable 1.8-3 package
% Mon Jan 21 23:48:21 2019
\begin{table}[htbp!]
\centering
\caption{F-test Reg3a} 
\label{tab:ftest1}
\begin{tabular}{lrrrrrr}
  \hline
 & Res.Df & RSS & Df & Sum of Sq & F & Pr($>$F) \\ 
  \hline
1 & 4432 & 524.31 &  &  &  &  \\ 
  2 & 4428 & 449.29 & 4 & 75.03 & 184.86 & 0.0000 \\ 
   \hline
\end{tabular}
\end{table}

% latex table generated in R 3.5.1 by xtable 1.8-3 package
% Mon Jan 21 23:48:21 2019
\begin{table}[htbp!]
\centering
\caption{F-test Reg3b} 
\label{tab:ftest2}
\begin{tabular}{lrrrrrr}
  \hline
 & Res.Df & RSS & Df & Sum of Sq & F & Pr($>$F) \\ 
  \hline
1 & 4432 & 529.46 &  &  &  &  \\ 
  2 & 4428 & 367.22 & 4 & 162.24 & 489.09 & 0.0000 \\ 
   \hline
\end{tabular}
\end{table}

% latex table generated in R 3.5.1 by xtable 1.8-3 package
% Wed Jan 16 16:54:16 2019
\begin{table}[htbp!]
\centering
\caption{F-test Reg3a non-overlapping sample} 
\label{tab:ftestOverlap1}
\begin{tabular}{lrrrrrr}
  \hline
 & Res.Df & RSS & Df & Sum of Sq & F & Pr($>$F) \\ 
  \hline
1 & 4432 & 524.31 &  &  &  &  \\ 
  2 & 4428 & 449.29 & 4 & 75.03 & 184.86 & 0.0000 \\ 
   \hline
\end{tabular}
\end{table}

% latex table generated in R 3.5.1 by xtable 1.8-3 package
% Wed Jan 16 11:15:11 2019
\begin{table}[htbp!]
\centering
\caption{F-test Reg3b non-overlapping sample} 
\begin{tabular}{lrrrrrr}
  \hline
 & Res.Df & RSS & Df & Sum of Sq & F & Pr($>$F) \\ 
  \hline
1 & 4432 & 529.46 &  &  &  &  \\ 
  2 & 4428 & 367.22 & 4 & 162.24 & 489.09 & 0.0000 \\ 
   \hline
\end{tabular}
\end{table}

%\restoregeometry
%\clearpage




	\newpage
	 % die entsprechenden Teile müssen ebenfalls im entsprechenden Ordner vorhanden sein. Jeder Teil muss mit dem Magic Comment %!TEX root = ../Main.tex  beginnen

	\newpage
	\pagenumbering{Roman}
	\printbibliography


%!TEX root = ../Main.tex

\section*{Declaration of Authorship}

I hereby declare that the paper \\
\begin{itshape} ``The Information Content of VIX Volatility" \end{itshape}\\
is my own unaided work. All direct or indirect sources used are acknowledged as references. I am aware that the paper in digital form can be examined for the use of unauthorized aid and in order to determine whether the thesis as a whole or parts incorporated in it may be deemed as plagiarism. This paper was not previously presented to another examination board and has not been published.

\vspace{3 cm}

\begin{tabular}{ll}
\makebox[0.3\textwidth]{\hrulefill} & \makebox[0.6\textwidth]{\hrulefill}\\
Place, Date & author's signature\\
\\
\end{tabular}
 
\end{document}
