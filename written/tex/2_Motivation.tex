%!TEX root = ../Main.tex

\section{Selected models of volatility measurement}

There are different models for the measurement and estimation of volatility. According to \citeauthor{andersen2001} then shall be grouped in \begin{quote}
(i) the notional volatility corresponding to the ex-post sample-path return variability over a fixed time interval, (ii) the ex-ante expected volatility over a fixed time interval or the (iii) the instantaneous volatility corresponding to the strength of the volatility process at a point in time.
\end{quote}
After a short section about the stylized facts of financial data, the different methods should be quickly presented, including how well they fit to the stylized facts. This leads to the conclusion, why HAR-RV is the model that should be used for this work.


\subsection{Stylized Facts of Financial Data}

\citeauthor{andersen2001} found three stylized facts for the spot exchange rate market. First even though raw returns have a leptocurtic distribution, the returns standardized by realized volatility are approximately Gaussian. Second, the distribution of realized volatility of returns itself is right skewed, the one of the logarithms of realized volatility however are also approximately Gaussian. Third, the long-run dynamics of realized logarithmic volatilities are well approximated by a fractionally-integrated long-memory process. 

Financial asset return volatility is time-varying,  not only across time-periods, but also across asset classes, assets and countries \parencite{andersen2001}.

This stylized facts motivate the use of the HAR-RV model.

\begin{itemize}
\item strong persistence of the autocorrelation of square and absolute returns \parencite{jiang2003}
\item return distributions exhibits both fat tails and tail crossover \parencite{jiang2003}
\item 
\end{itemize}

\subsection{Models Using Historic Volatility}
\subsubsection{The General Idea of Historic Volatility}
\begin{itemize}\itemsep0pt
\item historic volatility: just lagged realized volatility
\item econometric models: GARCH (as an example for a frequently used one)
\end{itemize}	

\subsubsection{HAR-RV Model}

Also \citeauthor{andersen2003} point out the advantage of using high-frequency returns is not only that they help predicting again high-frequency returns, but also that they contain information for longer horizons, such as monthly or quarterly. 


\subsection{Implied volatility}
\subsubsection{Black-and-Scholes Implied Volatility}
\begin{itemize}\itemsep0pt
\item explain basic idea of BS implied volatility
\item advantages of BS implied volatility: forward-looking nature of option prices
\item disadvantages of BS implied volatility: joint hypothesis problem due to underlying  pricing assumption (is a joint test of market efficiency and underlying pricing assumption), use only at-the-money options and fail to incorporate information,..
\end{itemize}

\textcolor{gray}{
Disadvantages of Black and Scholes: Black and Scholes uses only at-the-money option and thus fails to incorporate information \parencite{jiang2003}.
Black and Scholes are joint tests of market efficiency and the B-S model, thus studies are subject to model misspecification errors \parencite{jiang2003}.}

\subsubsection{VIX and Model-Free Implied Volatility}
\begin{itemize}\itemsep0pt
\item explain basic idea of model-free implied volatility
\item advantages of model-free implied volatility: solved joint hypothesis problem (direct test of market efficiency), can incorporate not only at-the-money options,..
\item the VIX as the model-free implied volatility estimate from the Cboe 
\end{itemize}

\textcolor{gray}{
Primilary described and derived by \citeauthor{britten2000}. Instead of being based on a specific option pricing model, it is derived entirely from no-arbitrage conditions. After that some papers did various corrections, such as \citeauthor{jiang2003} extended the model so that is not derived under diffusion assumptions and generalized it to processes including random jumps. Two advantage of the model-free option implied volatility, are firstly that it has no pricing assumption and thus constitutes a direct test of the option market's informational efficiency, and not a joined test of market efficiency and an assumed option pricing model. Secondly it incorporates information from options across different strike prices. }
