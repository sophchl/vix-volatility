%!TEX root = ../Main.tex

\section{Selected volatility concepts and models of volatility measurement}
This section presents first some stylized facts of financial data, and gives an introduction to the different ways to estimate volatility. By pointing out the advantages and disadvantages of the concepts and models and their fit to the stylized facts, the HAR-RV approach shall be motivated. 


\subsection{The Return Process and Stylized Facts of Financial Data}
As mentioned in the introduction, the challenge when measuring volatility is, that stock return volatility is not directly observable \parencite{tsay2005}. This problem evolves from the fact that we can only observe one realization of the underlying data generating process, and even though stocks are traded and thus have market prices which could be used for volatility measurement, there is no continuous data available and even for high-frequency data and extremely liquid markets microstructure effects and noise prevent getting close to a continuous sample path. It is thus only possible to estimate avearages of discrete volatility for a given period of time. \parencite{andersen2001}.\\
There are however several approaches that should be introduced here. To start with, the definition of the simple gross return is
\begin{align}\label{eq:return}
1+ R_{t} = \frac{P_{t}}{P_{t-1}} 
\end{align}
In the continuous-time setting, continuously compounded returns are used, which are given by
\begin{align*}
r_{t} = ln(1 + R_{t}) = ln (\frac{P_{t}}{P_{t}} + \frac{P_{t-1} - P_{t}}{P_{t}}) = 
ln \frac{P_{t}}{P_{t-1}} = p_{t} - p_{t-1} \ 
with\  p_{t} = ln(P_{t})
\end{align*}

When observed over time, this asset returns show some distributional properties, often referred to as stylized facts of asset returns. Observed by many authors, only a few shall be mentioned here. \citeauthor{corsi2009} for example mentions particularly the very strong persistence of autocorrelation of the square and absolute returns, which regularly poses challenges to econometric models. Moreover return probability density functions are often leptocurtic with fat tails. As the time scale increases, the return distribution slowly converge to the normal distribution, but before convergence, the return distribution has different shapes depending on the time scale. Financial data also show evidence of scaling as described firstly by Mandelbrot, which is connected to the idea that patterns appear in different times, or that the distribution for returns has similar functional forms for various choices of the time interval. \\
Moreover \citeauthor{andersen2001} mentions that first, even though raw returns have a leptocurtic distribution, the returns standardized by realized volatility are approximately Gaussian. Second, the distribution of realized volatility of returns itself is right skewed, the one of the logarithms of realized volatility however are also approximately Gaussian. Third, the long-run dynamics of realized logarithmic volatilities are well approximated by a fractionally-integrated long-memory process. Other authors who mention stylized facts: \parencite{jiang2003}.

\subsection{Concepts and Models using Historic Volatility}
\subsubsection{Volatility Concept - Realized Volatility}
By definition ``volatility seeks to capture the strength of the (unexpected) return variation over a given period of time'' \parencite[p.7]{andersen2001}. However, there are multiple concepts and definitions of asset volatility. According to \citeauthor{andersen2001} the concepts can be grouped in (i) the \emph{notional volatility} corresponding to the ex-post sample-path return variability over a fixed time interval, (ii) the ex-ante \emph{expected volatility} over a fixed time interval or the (iii) the \emph{instantaneous volatility} corresponding to the strength of the volatility process at a point in time.
For this paper, given the dataset of actual return observations, one can compute the ex-post realized volatility.\\
It can be shown, that under some assumptions, realized volatility as the sum of squared high frequency returns, can be used to approximate the quadratic variation process which is the variation in a continuous time setting. This approach mainly building on the work of \citeauthor{andersen2001} und [noch jemanden finden] shall only be briefly introduced here. \\
To begin with, it should be assumed that we have a continuous-time no-arbitrage setting. As return volatility aims to capture the strength of the unexpected return variation, one needs to define the component of a price change as opposed to an expected price movement. In discrete time this can be done by specifying the conditional mean return using for example an asset pricing model. In the continuous time setting however it requires the decomposition of the return process in an expected and innovation component. \citeauthor{andersen2001} show, that under certain assumptions the log-price process must constitute a semi-martingale process, which allows for the decomposition of the instantaneous return process into an expected return component, and a martingale innovation. To estimate volatility, \citeauthor{andersen2001} show furthermore, that one can refer to the quadratic variation process of this martingale component for volatility measurement, as the quadratic variation process represents the (cumulative) realized sample path variability of the martingale over any fixed interval \parencite{andersen2001}. For the purpose of this paper it shall only be said, that quadratic variation is... . To be precise, they define the notional (ex-post) volatility as the increment to the quadratic variation for the return series. They also state, that this increment to quadratic variation can be consistently and well approximated through the accumulation of high-frequency squared returns.


For this paper, this approximation of notional volatility shall be termed \emph{realized volatility}.



\subsubsection{Volatility Model - HAR-RV Model}

To measure volatility, one can separate between parametric and non-parametric methods, where parametric models are both discrete and continuous time methods. For an encompassing overview, please see \citeauthor{andersen2001}. 

Also \citeauthor{andersen2003} point out the advantage of using high-frequency returns is not only that they help predicting again high-frequency returns, but also that they contain information for longer horizons, such as monthly or quarterly. 


\subsection{Implied volatility}
\subsubsection{The General Idea of Implied Volatility}
\begin{itemize}\itemsep0pt
\item explain basic idea of BS implied volatility
\item advantages of BS implied volatility: forward-looking nature of option prices
\item disadvantages of BS implied volatility: joint hypothesis problem due to underlying  pricing assumption (is a joint test of market efficiency and underlying pricing assumption), use only at-the-money options and fail to incorporate information,..
\end{itemize}

\textcolor{gray}{
Disadvantages of Black and Scholes: Black and Scholes uses only at-the-money option and thus fails to incorporate information \parencite{jiang2003}.
Black and Scholes are joint tests of market efficiency and the B-S model, thus studies are subject to model misspecification errors \parencite{jiang2003}.}

\subsubsection{VIX and Model-Free Implied Volatility}
\begin{itemize}\itemsep0pt
\item explain basic idea of model-free implied volatility
\item advantages of model-free implied volatility: solved joint hypothesis problem (direct test of market efficiency), can incorporate not only at-the-money options,..
\item the VIX as the model-free implied volatility estimate from the Cboe 
\end{itemize}

\textcolor{gray}{
Primilary described and derived by \citeauthor{britten2000}. Instead of being based on a specific option pricing model, it is derived entirely from no-arbitrage conditions. After that some papers did various corrections, such as \citeauthor{jiang2003} extended the model so that is not derived under diffusion assumptions and generalized it to processes including random jumps. Two advantage of the model-free option implied volatility, are firstly that it has no pricing assumption and thus constitutes a direct test of the option market's informational efficiency, and not a joined test of market efficiency and an assumed option pricing model. Secondly it incorporates information from options across different strike prices. }
