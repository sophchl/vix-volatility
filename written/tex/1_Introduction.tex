%!TEX root = ../Main.tex

\section{Introduction: The importance of volatility measurement}
\subsection{Why volatility matters: Volatility as key input to both risk measures and option pricing models}
\begin{itemize}\itemsep0pt
\item key input to risk measures (VaR etc)
\item key input to pricing derivative securities
\item financial market stability
\item Problems: can not be directly observed and thus has to be estimated 
\end{itemize}

\subsection{Weaknesses of existing models: VIX introduced by CBOE}
\begin{itemize}\itemsep0pt
\item power of volatility models lies in out-of-sample forecasting power
\item so far BS implied volatility models had the best out of sampling forecasting power, but they have several problems (most importantly joint hypothesis problem) 
\end{itemize}

\textcolor{gray}{
Volatility is for example both a key input factor to risk measures (such as Value-at-risk), or pricing of derivative securities, which both again are crucial for financial decision making. As volatility can not be observed as directly as price can, it has to be both estimated and forecasted. There are multiple ways to forecast volatility. The strenght of a volatility model however lies in it's out-of-sample forecasting power \parencite{poon2003}. \\
Volatility measures play an important role for financial market stability.\\
Stylized facts of financial market data suggests that return distributions are not i.i.d., meaning that the variance of returns over a long horizon can not be derived from a single observed period \parencite{poon2003}. }
