%!TEX root = ../Main.tex

\section{Results}\label{sec:5Results}

This section presents the estimation results. The first subsection reports the results obtained with the regression analysis described in section \ref{sec:42Method} and the second subsection provides the results of the robustness checks conducted. 

\subsection{Regression Analysis Results}\label{sec:51Regression}
The regression results can be found in Table \ref{tab:newey2} for the logarithmic regression and in Table \ref{tab:newey1} for the level regression, in the appendix. To make it easier to overview the results, only one regression table was included in this section. The logarithmic table is displayed, since as discussed in section \ref{sec:41Data} it should be better specified, which is confirmed by it having a lower AIC than the level specification.\\
Firstly, if the VIX contains information about one-day ahead realized volatility, the slope estimate for the VIX in \ref{eq:Reg2a} and \ref{eq:Reg2b} should be positive and significantly different from zero. This is the case for both specifications, the $H_{0}: \beta^{VIX} = 0$ can be rejected for both the level and the logarithmic regression. This is in alignment with \textbf{\ac{H1}}.\\
Secondly, concerning \textbf{\ac{H2}}, if the VIX has more explanatory power than the historic volatilities, the adjusted $R^{2}$ in the second regression with the \ac{VIX} (\ref{eq:Reg2a} and \ref{eq:Reg2b}) should be larger than in the first regression with the historic volatilities (\ref{eq:Reg1a} and \ref{eq:Reg1b}). This not true for both the level and the logarithmic specification. Even though the $R^{2}$'s are both high and close, it is slightly larger in the first regression ($0.726$) than in the second ($0.693$).\\
Thirdly, if the VIX adds explanatory value to the historic volatilities, the adjusted $R^{2}$ in the regression with only the VIX included (\ref{eq:Reg3a} and \ref{eq:Reg3b}) should be larger than in the regression containing only the historic volatilities (\ref{eq:Reg1a} and \ref{eq:Reg1b}). This is true for both specifications, in alignment with \textbf{\ac{H3}}. Actually, the full model is the one with the highest (adjusted) $R^{2}$ in both specifications. \\
Finally, if the VIX subsumes all information contained in the historic volatilities, the historic volatilities should not be significantly different from zero in the regression containing all explanatory variables, whereas the VIX should be significant. Even though the estimates for all historic volatilities decrease when the VIX is added only the monthly historic volatility turns insignificant in the logarithmic specification. This is confirmed by the results from the F-test, which can be found in the appendix in table \ref{tab:ftest1} and table \ref{tab:ftest2}. The $H_{0}: \beta^{d} = \beta^{w} = \beta^{m} = 0 \ and \ \beta^{VIX} = 1$ can be rejected at the $0.001$ significance level for both specifications. Thus, \textbf{\ac{H4}} can not be confirmed, it does not seem that the VIX subsumes the information contained in the three historic volatilities.\\
The estimate for the crisis dummy is also significantly different from zero in every regression in both the level and the logarithmic specification, thus including the crisis was reasonable. This confirms \textcite{jiang2003}, who describe that it is critical to include periods with particular financial turmoil, such as the regime shift  around the October 1987 crash.

%

% Table created by stargazer v.5.2.2 by Marek Hlavac, Harvard University. E-mail: hlavac at fas.harvard.edu
% Date and time: So, Jan 20, 2019 - 01:19:26
\begin{table}[!htbp] \centering 
\begin{threeparttable}
  \caption{Logarithmic regression (whole sample)} 
  \label{tab:newey2} 
\begin{tabular}{@{\extracolsep{5pt}}lccc} 
\\[-1.8ex]\hline 
\hline \\[-1.8ex] 
 & \multicolumn{3}{c}{\textit{Dependent variable:}} \\ 
\cline{2-4} 
\\[-1.8ex] & \multicolumn{3}{c}{Realized Volatility} \\ 
 & Reg1b & Reg2b & Reg3b \\ 
\\[-1.8ex] & (1) & (2) & (3)\\ 
\hline \\[-1.8ex] 
 Intercept & $-$0.043$^{***}$ & $-$0.407$^{***}$ & $-$0.187$^{***}$ \\ 
  & (0.007) & (0.017) & (0.014) \\ 
  & & & \\ 
 $ln(RV^{(d)}_{t})$ & 0.344$^{***}$ &  & 0.263$^{***}$ \\ 
  & (0.027) &  & (0.025) \\ 
  & & & \\ 
 $ln(RV^{(w)}_{t})$ & 0.395$^{***}$ &  & 0.284$^{***}$ \\ 
  & (0.035) &  & (0.035) \\ 
  & & & \\ 
 $ ln(RV^{(m)}_{t})$ & 0.208$^{***}$ &  & 0.014 \\ 
  & (0.024) &  & (0.029) \\ 
  & & & \\ 
 $crisis$ & 0.020$^{*}$ & $-$0.224$^{***}$ & $-$0.100$^{***}$ \\ 
  & (0.012) & (0.034) & (0.016) \\ 
  & & & \\ 
 $ln(VIX_{t})$ &  & 1.472$^{***}$ & 0.648$^{***}$ \\ 
  &  & (0.048) & (0.045) \\ 
  & & & \\ 
\hline \\[-1.8ex] 
AIC & 1874.2 & 2363.6 & 1551.6 \\ 
Observations & 4,434 & 4,434 & 4,434 \\ 
R$^{2}$ & 0.726 & 0.694 & 0.745 \\ 
Adjusted R$^{2}$ & 0.726 & 0.693 & 0.745 \\ 
Residual Std. Error & 0.299 (df = 4429) & 0.316 (df = 4431) & 0.288 (df = 4428) \\ 
\hline 
\hline \\[-1.8ex] 
\textit{Note:}  & \multicolumn{3}{r}{$^{*}$p$<$0.1; $^{**}$p$<$0.05; $^{***}$p$<$0.01} \\ 
\end{tabular} 
\begin{tablenotes}
\small
\item  The numbers in the brackets are the standard errors of the parameters computed with Newey-West covariance correction, which are robust to autocorrelated and heteroscedastic error terms, see \textcite{newey1987}.
\end{tablenotes}
\end{threeparttable}
\end{table} 


\subsection{Robustness Checks}\label{sec51Robustness}

Early studies (for example \textcite{canina1993}) testing the information content of (model-free) implied volatility often used overlapping samples, meaning that the same option is used in several implied-volatility calculations. However, \textcite{christensen2001} showed, that the use of overlapping samples creates a telescopic overlap problem under which standard statistical inferences (including t-statistics and $R^{2}$s) are no longer valid.\\
Therefore the same regression analysis was conducted using non-overlapping samples. \textcite{jiang2003} use monthly non-overlapping samples, using the first Wednesday of every month, since they calculate the implied volatility over a horizon of one month. The VIX however is calculated slightly differently. It contains near- and next-term options options between 23 and 37 days to maturity (which is always a Friday), and every week the options roll over to new maturities. This shall be illustrated taking the example from \textcite{exchange2009}. If for example the second Tuesday in October is taken, the near-term option expires in 24 days, and the next-term option in 31 days. One day later, the option that expires now in 30 days is the near-term option, and another option expiring in 37 days is the next-term option. This next-term option will, one week later, roll over to a near-term option and, one more week later, drop out of the calculation. Thus, an option can be included in the calculation for up to two weeks. \\
To avoid, that an option can appear in two volatilities, after the volatilities and their lags are calculated, only one value every two weeks is used. As in \textcite{jiang2003}, the values of Wednesday are used, having only few holiday days also in the sample of this paper, for each second week.\\
The estimation results for the sample using non-overlapping data are summarized in Table \ref{tab:overlap1} and Table \ref{tab:overlap2} in the appendix.\\
The regression results using non-overlapping samples confirm all the results described for the regressions using the full sample. Whereas \textbf{\ac{H1}}, that the VIX contains significant information about future volatility and \textbf{\ac{H3}}, that the VIX adds information content to the historic volatilities can be confirmed, whereas \textbf{\ac{H2}}, that the VIX has more explanatory power than the historic volatilities and \textbf{\ac{H4}}, that the VIX contains all information contained in the historic volatilities cannot be confirmed. Concerning the qualities and predictive power of the model, also for this sample the AIC is lower in the logarithmic specification and the adjusted $R^{2}$, even though high in both specifications ($0.796$ and $0.770$), is higher in the logarithmic regression. Within the specifications, it is also always the full model with the highest information content for one-day ahead realized volatility.\\
Nevertheless, there are two differences. Firstly, in the full-sample regression all estimates for the historic volatilities decreased when the VIX was added, and the monthly historic volatility turned insignificant only in the logarithmic specification. In the non-overlapping sample regression all estimates of historic volatilities decrease, too, but the monthly historic volatility turned insignificant in both the level and the logarithmic specification. Thus, apparently the VIX subsumes this information. Secondly, both the intercept and the estimate for the crisis dummy are insignificant in the regression containing only the historic volatilities.


\newpage
















