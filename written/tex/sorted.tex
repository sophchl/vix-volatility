\begin{itemize}\itemsep0pt
\item key input to risk measures (VaR etc)
\item key input to pricing derivative securities
\item financial market stability
\item Problems: can not be directly observed and thus has to be estimated  
\end{itemize}

THE JOINT DISTRIBUTIONAL characteristics of asset returns are pivotal for many
issues in financial economics. They are the \textbf{key ingredients for the pricing of
financial instruments}, and they speak directly to the risk-return tradeoff central
to portfolio allocation, performance evaluation, and managerial decision-making. 
fractiles of conditional portfolio
return distributions, which govern the likelihood of extreme shifts in portfolio
value and are therefore central to financial risk management, figuring promi-nently in both regulatory and private-sector initiatives. The most critical feature of the conditional return distribution is arguably itssecond moment structure, which is empirically the dominant time-varying char-acteristic of the distribution. \parencite{andersen2018}.


\textcolor{gray}{
Volatility is for example both a key input factor to risk measures (such as Value-at-risk), or pricing of derivative securities, which both again are crucial for financial decision making. As volatility can not be observed as directly as price can, it has to be both estimated and forecasted. There are multiple ways to forecast volatility. The strenght of a volatility model however lies in it's out-of-sample forecasting power \parencite{poon2003}. \\
Volatility measures play an important role for financial market stability.\\
Stylized facts of financial market data suggests that return distributions are not i.i.d., meaning that the variance of returns over a long horizon can not be derived from a single observed period \parencite{poon2003}. }


A comfortable feature of the continuously compounded returns, is that they are time additive, meaning
\begin{align*}
r(t,h) = r(t) - r(t-h) \ with \ 0 \leq h \leq t \leq T.
\end{align*}
Assuming, that the asset prices are positive and finite, both price and return are defined in the interval [0,T] and as a consequence, r(t) has only countable many jump points in [0,T]. \\
Assuming furthermore that the return process is a càdlàg process, that there are no arbitrage opportunities and frictions and  that the expected return is finite, then the log-price process must constitute a semi-martingale. This leads to the following decomposition of the instantaneous return, into an expected return component and a martingale innovation
\begin{align*}
r(t) = p(t) - p(0) = \mu(t) + M(t) = \mu(t) + M^{c}(t) + M^{j}(t)
\end{align*}
where $\mu(t)$ is a predictable and finite variation process, $M(t)$ is a local martingale which may be further decomposed into $M^{c}(t)$, a continuous sample path, infinite variation local martingale component, and $M^{j}(t)$, a compensated jump martingale.\\
Unfortunately, instantaneous returns can not be observed, and even in liquid markets microstructure effects distort the observation of an even closely continuous sample-path realization. Consequently this decomposition has to be transferred to the discrete interval setting. This is slightly complex, and for this work shall only be constituted, that that in discrete time there are two distinct terms in the return innovation instead of one, however one of these terms is a martingale component, too, and this is the dominant part. 

\begin{figure}[!htbp]
\caption{Level-level regression}
\centering

\begin{tabular}{l c c }
\hline
 & Historic & Historic with VIX \\
\hline
Intercept    & $0.00^{***}$ & $-0.01^{***}$ \\
             & $(0.00)$     & $(0.00)$      \\
$RV_{t}^{d}$ & $0.27^{***}$ & $0.23^{***}$  \\
             & $(0.02)$     & $(0.02)$      \\
$RV_{t}^{w}$ & $0.39^{***}$ & $0.36^{***}$  \\
             & $(0.03)$     & $(0.03)$      \\
$RV_{t}^{m}$ & $0.25^{***}$ & $-0.04$       \\
             & $(0.03)$     & $(0.03)$      \\
VIX          &              & $0.00^{***}$  \\
             &              & $(0.00)$      \\
\hline
R$^2$        & 0.54         & 0.57          \\
Adj. R$^2$   & 0.54         & 0.57          \\
Num. obs.    & 4436         & 4436          \\
RMSE         & 0.02         & 0.02          \\
\hline
\multicolumn{3}{l}{\scriptsize{$^{***}p<0.001$, $^{**}p<0.01$, $^*p<0.05$}}
\end{tabular}

\end{figure}
%
\begin{figure}[!htbp]
\caption{log-log regression}
\centering

\begin{tabular}{l c c }
\hline
 & Historic & Historic and VIX \\
\hline
Intercept    & $-0.02^{**}$ & $-2.51^{***}$ \\
             & $(0.01)$     & $(0.09)$      \\
$RV_{t}^{d}$ & $0.34^{***}$ & $0.24^{***}$  \\
             & $(0.02)$     & $(0.02)$      \\
$RV_{t}^{w}$ & $0.40^{***}$ & $0.28^{***}$  \\
             & $(0.03)$     & $(0.03)$      \\
$RV_{t}^{m}$ & $0.21^{***}$ & $-0.13^{***}$ \\
             & $(0.02)$     & $(0.03)$      \\
VIX          &              & $0.80^{***}$  \\
             &              & $(0.03)$      \\
\hline
R$^2$        & 0.73         & 0.77          \\
Adj. R$^2$   & 0.73         & 0.77          \\
Num. obs.    & 4437         & 4437          \\
RMSE         & 0.30         & 0.28          \\
\hline
\multicolumn{3}{l}{\scriptsize{$^{***}p<0.001$, $^{**}p<0.01$, $^*p<0.05$}}
\end{tabular}

\end{figure}
%
% latex table generated in R 3.5.1 by xtable 1.8-3 package
% Wed Jan 02 18:28:25 2019
\begin{table}[ht]
\centering
\begin{tabular}{rrr}
  \hline
 & AIC & BIC \\ 
  \hline
OLS & 2819.78 & 2851.77 \\ 
  OLS with VIX & 2118.89 & 2157.27 \\ 
  log OLS & 1862.08 & 1894.07 \\ 
  log OLS with VIX & 1316.67 & 1355.05 \\ 
  HAR-RV & -56620.66 & -56588.65 \\ 
   \hline
\end{tabular}
\end{table}
