\begin{itemize}\itemsep0pt
\item key input to risk measures (VaR etc)
\item key input to pricing derivative securities
\item financial market stability
\item Problems: can not be directly observed and thus has to be estimated  
\end{itemize}

THE JOINT DISTRIBUTIONAL characteristics of asset returns are pivotal for many
issues in financial economics. They are the \textbf{key ingredients for the pricing of
financial instruments}, and they speak directly to the risk-return tradeoff central
to portfolio allocation, performance evaluation, and managerial decision-making. 
fractiles of conditional portfolio
return distributions, which govern the likelihood of extreme shifts in portfolio
value and are therefore central to financial risk management, figuring promi-nently in both regulatory and private-sector initiatives. The most critical feature of the conditional return distribution is arguably itssecond moment structure, which is empirically the dominant time-varying char-acteristic of the distribution. \parencite{andersen2018}.


\textcolor{gray}{
Volatility is for example both a key input factor to risk measures (such as Value-at-risk), or pricing of derivative securities, which both again are crucial for financial decision making. As volatility can not be observed as directly as price can, it has to be both estimated and forecasted. There are multiple ways to forecast volatility. The strenght of a volatility model however lies in it's out-of-sample forecasting power \parencite{poon2003}. \\
Volatility measures play an important role for financial market stability.\\
Stylized facts of financial market data suggests that return distributions are not i.i.d., meaning that the variance of returns over a long horizon can not be derived from a single observed period \parencite{poon2003}. }