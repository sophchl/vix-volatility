%!TEX root = ../Main.tex

\section{Methodology and Data}\label{sec:4MethodData}
The aim of this paper is to build on the results from previous research, for example \textcite{jiang2003}, who justify both the general validity of the model-free implied volatility and the decision of \ac{CBOE} to modify the calculation of the \ac{VIX} index. This paper will examine directly the information content of the \ac{VIX} index compared to historic volatility, using a more recent and longer time period of 17 years, between January 2000 and December 2017. Moreover, more information regarding the historic volatility is included with the use of the HAR-RV model by \textcite{corsi2009}. In this section, first the sample and the model is introduced, then the testable hypothesis are formalized. Finally the limitations of the methodology are briefly discussed.

\subsection{Data and Calculation of Input Factors}\label{sec:41Data}
% WHERE DOES THE DATA COME FROM %
The data used in this study is obtained from several sources. The realized variance of the \ac{SPX} is obtained from the \textcite{Oxford:RV}. Daily \ac{VIX} values are the closing values from \ac{CBOE}. Since September 22, 2003 \ac{CBOE} started to use the model-free implied volatility for the \ac{VIX}, but calculated the prices with the new methodology ex-post, dating back until 1990. As the sample from this study covers both the period before and after the revised methodology, the data was taken both from the ex-post \textcite{CBOE:old} and the daily-updated \textcite{CBOE:new}. The \ac{SPX} closing values, used only for visualisation, are also taken from \textcite{SandP}. The sample consists of daily data in the period from January 2000 to December 2017. \\
% CALCULATION OF INPUT FACTOR: REALIZED VOLATILITY %
The calculation of the daily realized variance is done by \textcite{Oxford:RV} using the sum of squared 5-min high-frequency returns, as explained in section \ref{sec:221RV}. The formula is given by 
\begin{align}
\sigma_{t} = \sum x_{t}^{2}
\end{align}
with $x_{t} = X_{t} - X_{t-1}$ and $X_{t}$ the logarithm of the price at time $t$. \\%
Previous research has used different sampling frequencies for the calculation of the realized volatility. Whereas earlier studies used mainly daily returns, more recent studies argue in favour of using intraday returns, as they have certain advantages over daily data \parencite{jiang2003}. For example \textcite{andersen2003} point out that high-frequency returns help both for predicting again high-frequency returns, but also that they contain information for longer horizons, such as monthly or quarterly. Moreover and more importantly, \textcite{andersen1998} show that realized volatility calculated using squared returns produces inaccuracies with daily returns. The realized variace data is cleaned by \textcite{Oxford:RV}. First, entries outside the timestamp when exchanges are open are deleted. Secondly, entries with the same time stamp are replaced with the median bid-ask price. Thirdly, entries with a negative spread (as they violate the no-arbitrage condition) or extremely large spread (50 times the median of the day) are deleted. Lastly, entries for which the mid-quote deviated largely from the mean are removed. For this paper, to obtain the realized volatility, the square root of the realized variance is taken. Moreover, for a more intuitively comparison to the \ac{VIX}, the values are multiplied by 100:
\begin{align}
RV_{t} = \sqrt{\sum x_{t}^{2}} \times 100.
\end{align}
% CALCULATION OF THE INPUT FACTOR: HISTORIC VOLATILITY %
For the historic volatility the lagged realized volatility is used. As the approach from the HAR-RV model presented in \ref{sec:222HAR-RV} is used, additionally to lagged daily realized volatility, the lagged weekly and lagged monthly volatility is needed. They are computed using the rolling average over the respective time periods, thus weekly realized volatility over the last 5 trading days is calculated:
\begin{align}
RV_{t}^{(w)} = \frac{1}{5} (RV_{t}^{(d)} + RV_{t-1d}^{(d)} + ... + RV_{t-4d}^{(d)}).
\end{align}
The monthly volatility is calculated alike, using a period of 20 days.\\
% CALCULATION OF INPUT FACTOR: VIX %
For the \ac{VIX} the daily closing value is taken, as it contains the information from the whole day. The calculation is described in section \ref{sec:223VIX}. To present the data in a more intuitive manner, the annualized \ac{VIX} is divided by $\sqrt{252}$ (as for example in \textcite{blair2001} and \textcite{whaley2008}).\\
% DESCRIPTIVE STATISTICS %
The \ac{SPX} together with the \ac{VIX} is illustrated in figure \ref{fig:SPandVIX} and the realized volatility is added in figure \ref{fig:SPandVIXandVol} in the appendix. The summary statistics can be found in Table \ref{tab:summary1} and using logarithm in Table \ref{tab:summary2}, equally in the appendix. The summary statistics and figures show, that the \ac{VIX} is for every time period slightly higher than the realized variance. This is consistent with the findings previous research, e.g. \textcite{jiang2003}. However, the graphics show, that during crisis periods, the realized volatility exceeds the \ac{VIX}. For example, whereas both the realized volatility and the \ac{VIX} are particularly high during the crisis around 2008, the realized variance is peaking even higher. To account for this period, a dummy for the crisis 2008 - 2012 was included in the model.\\
Moreover the summary statistics show, that the skewness and kurtosis of the logarithmic specification is closer to the one of the normal distribution. Consequently, a regression based on the log volatilities is also specified and should be statistically better specified than those based on simple volatility.\\
The correlation matrix for the realized volatility, its lagged specifications and the model-free implied volatility and its lags can be found in Table \ref{tab:correlation} in the appendix. Overall, the realized volatility is highly correlated with both the past realized volatility and the past \ac{VIX} values. For comparison, also the weekly and monthly \ac{VIX} is included, calculated in the same way as the weekly and monthly realized volatility. Whereas for the one-day lag the correlation of the realized volatility with the \ac{VIX} is higher, for the lagged weekly and monthly averages the correlation with the historic volatilities is higher. 

\subsection{Methodology: Linear Regression and HAR-RV model}\label{sec:42Method}
Consistent with prior research, for example \textcite{jiang2003}, \textcite{canina1993} or \textcite{christensen1998}, both univariate and encompassing regression analysis is used to analyse the information content of the volatility measures. However, additionally the approach from the HAR-RV model, described by \textcite{corsi2009} is integrated. This means, not only one day lagged realized volatility is used as an explanatory variable in the regression, but also weekly and monthly realized volatility, computed as described in section \ref{sec:41Data}.\\
In the univariate regression, the realized volatility is regressed once solely on the historic data, and once only on the \ac{VIX}. For comparison, realized volatility is regressed on both historic data and the \ac{VIX} in the encompassing regression analysis. Thus the encompassing regression analysis gives information about the relative importance of the volatility measures, and whether the \ac{VIX} subsumes the information from the historic volatility \parencite{jiang2003}. For each explanatory variable, the value of the current day $t$ is used, whereas for the explained variable, the one-day ahead value $t+1d$ is used. Like this, all the information available on day $t$ is used to evaluate the one-day ahead realized volatility. The three regressions are then given by
\begin{align}
RV_{t+1d} = \ &c + \beta^{RV,d}_{t} RV^{(d)}_{t} + \beta^{RV,w}_{t} RV^{(w)}_{t} + \beta^{RV,m}_{t} RV^{(m)}_{t} + \beta^{crisis} crisis \tag{Reg1a}  \label{eq:Reg1a}\\
RV_{t+1d} = \ &c + \beta^{VIX}_{t} VIX_{t} + \beta^{crisis} crisis  \tag{Reg2a}  \label{eq:Reg2a}\\
RV_{t+1d} = \ &c + \beta^{RV,d}_{t} RV^{(d)}_{t} + \beta^{RV,w}_{t} RV^{(w)}_{t} + \beta^{RV,m}_{t} RV^{(m)}_{t} + \beta^{VIX}_{t} VIX_{t}  \nonumber \\
& + \beta^{crisis} crisis  \tag{Reg3a}  \label{eq:Reg3a}
\end{align}
with $RV_{t+1d}$ the realized volatility one-day-ahead, $RV^{(d)}_{t}$ the daily realized volatility, $RV^{(w)}_{t}$ the weekly realized volatility, $RV^{(m)}_{t}$ the monthly realized volatility, $VIX$ the \ac{VIX} closing value and $crisis$ the dummy variable, indicating one in the time of the financial crisis (2008-2012) and zero otherwise. The same regressions are specified with the logarithm for each variable,
\begin{align}
\ln(RV_{t+1d}) = \ &c + \beta^{RV,d}_{t} \ln(RV^{(d)}_{t}) + \beta^{RV,w}_{t} \ln(RV^{(w)}_{t}) \nonumber \\
& + \beta^{RV,m}_{t} \ln(RV^{(m)}_{t}) + \beta^{crisis} crisis  \tag{Reg1b}  \label{eq:Reg1b}\\
ln(RV_{t+1d}) =  \ &c + \beta^{VIX}_{t} ln(VIX_{t}) + \beta^{crisis} crisis  \tag{Reg2b}  \label{eq:Reg2b}\\
ln(RV_{t+1d}) = \ &c + \beta^{RV,d}_{t} ln(RV^{(d)}_{t}) + \beta^{RV,w}_{t} ln(RV^{(w)}_{t}) + \beta^{RV,m}_{t} ln(RV^{(m)}_{t}) \nonumber \\
& + \beta^{VIX}_{t} ln(VIX_{t})+ \beta^{crisis} crisis.  \tag{Reg3b}  \label{eq:Reg3b}
\end{align}
In all regression specifications the Newey-West covariance correction is used, to account for the possible presence of serial correlation in the data.

\subsection{Research Questions}\label{sec:41aHypothesis}
Model-free implied volatility, contrary to historic volatility, uses the forward-looking nature of options and does not suffer from misspecification problems. Moreover, it aggregates information across all strike prices. It is therefore assumed, that it is informationally more efficient than historic volatility. \ac{BS} implied volatility is not included, as previous research found that model-free implied volatility outperforms \ac{BS} implied volatility (e.g. \textcite{jiang2003, bakanova2010}).\\
The regressions will be examined with the following research questions :\\
\textbf{Q1:} Does the \ac{VIX} contain information about one-day ahead realized volatility?\\
\textbf{Q2:} Does the \ac{VIX} have more explanatory value than the historic volatilities in estimating one-day ahead realized volatility?\\
\textbf{Q3:} Does \ac{VIX} add explanatory value to the historic volatilities in estimating one-day ahead realized volatility?\\
\textbf{Q4:} Does the \ac{VIX} incorporate all information regarding one-day ahead realized volatility, the historic volatilities contain no information beyond what is already included in the \ac{VIX}?\\
The research questions are approached using the regressions in the following way:\\
Concerning \textbf{Q1}, if the \ac{VIX} contains information about future volatility, we would expect the slope estimate of the VIX in \ref{eq:Reg2a} and \ref{eq:Reg2b} to be significantly different from zero. This can formalized as a testable hypothesis, with $H_{0}: \beta^{VIX} = 0$, using t-tests. For \textbf{Q2}, adjusted $R^{2}$ should be compared in \ref{eq:Reg1a} to \ref{eq:Reg2a} and \ref{eq:Reg1b} to \ref{eq:Reg2b}. If the \ac{VIX} can explain more variation of the one-day-ahead realized volatility than the historic volatilities, the $R^{2}$ and adjusted $R^{2}$ should be larger in the second regression\footnote{For this paper the adjusted $R^{2}$ is preferred to the normal $R^{2}$, as it adds a penalty for adding additional independent variables, which could be misleading here because the first specification has more explanatory variables than the second one.}. For \textbf{Q3}, adjusted $R^{2}$ should be compared in \ref{eq:Reg2a} to \ref{eq:Reg3a} and \ref{eq:Reg2b} to \ref{eq:Reg3b}. If the \ac{VIX} adds explanatory value additional to the historic volatilities, the (adjusted) $R^{2}$ should increase in the third specification. Finally, for \textbf{Q4}, if the VIX subsumes the information content contained in the historic volatilities, we would expect the slope estimates for the historic volatilities to become not significantly different from zero, when the VIX is added, whereas the VIX should still be significant (\ref{eq:Reg3a} and \ref{eq:Reg3b}). This can be formalized as a testable hypothesis with the $H_{0}: \beta^{d} = \beta^{w} = \beta^{m} = 0 \ and \ \beta^{VIX} = 1$, using F-tests.

\subsection{Limitations}\label{sec:43Limits}
As volatility is stochastic, the ex-ante estimation will not equal the return volatility, as it is a measurement over an aggregated discrete time period \parencite{andersen2001}.
The VIX might be flawed, as \textcite{jiang2007} showed.
There might be a bias in estimated realized volatility due to autocorrelation in intraday returns \parencite{jiang2003}. 

%  as serial correlation causes both the Gauss-Markow and Classical linear model assumptions to fail. 