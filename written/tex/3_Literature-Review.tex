%!TEX root = ../Main.tex

\section{Empirical Results on the Information Content of Model-Free Implied Volatility and Hypothesis}\label{sec:3Literature}
Various papers tested the informational efficiency of the mode-free implied volatility, some using the VIX. This section shall give a brief overview of the existing literature and introduce the hypothesis for this paper.

\subsection{Review of Empirical Results on Measuring the Information Content of Model-Free Implied Volatility}\label{sec:31LiteratureResults}

Firstly, there are numerious papers testing the informational efficiency of the VIX before 2003, for example \textcite{blair2001}. These papers are however not included in this literature review section, as they do not use the currently applied model-free implied volatility version of the VIX, which was only introduced in 2003. \\
An early paper that examines the information content of model-free implied volatility is \textcite{jiang2003}. They use the approach of \textcite{britten2000}, extend the formula to asset price processes with jumps and test the informational efficiency of this model-free implied volatility in comparison to both \gls{BS} implied and historic volatility, using both univariate and encompassing regression analysis. Using option data from the S\&P 500 for the model-free implied volatility calculation and 5-min returns to calculate daily realized volatility, they examine monthly non-overlapping samples of a 6-year sample period between June 1988 and December 1994. Their findings are, that the model-free implied volatility subsumes all the information contained in the \gls{BS} implied volatility and past realized volatility. Another example is the paper of \textcite{bakanove2010}. With daily data from oil futures between November 1986 and December 2006, they evaluate the information content of model-free implied volatility in comparison to historic volatility using a monthly frequency. 
Using regression analysis, they come to the result that implied volatility subsumes the information contained in historical volatility. In contrast to these results is for example \textcite{taylor2010}. With individual stock data from 149 U.S. firms between January 1996 and December 1999, they find that for a one-day-ahead estimation, historic volatility outperforms model-free implied volatility. \\
Apart from the papers examining informational efficeiency, there are  papers using and extending the implied-volatility approach are for ecample \textcite{hao2013}, who extend the VIX by deriving the VIX formulas under a risk-neutral valuation relationship, finding that this GARCH implied volatility is significantly lower than the VIX thus interpreting that the GARCH model can not capture the variance premium. \\


\subsection{Research Question and Hypothesis}\label{sec:32Hypothesis}
The aim of this paper is to use the results from \textcite{jiang2003}, who use their results to justify both the general validity of the model-free implied volatility, and the decision of \gls{CBOE} to modify the calculation of the VIX index. Thus this paper will examine directly the information content of the VIX index, using a more recent and longer time period of 17 years, between January 2000 and December 2017. 
Moreover, the approach from the HAR-RV model as described by \textcite{Corsi2009} is used.\\
Model-free implied volatility uses the forward-looking nature of options in comparison to historic volatility and does not suffer from the misspecification problems that occur when testing \gls{BS} implied volatility,moreover it aggregates information across all strike prices. It is therefore assumed, that it is informationally more efficient than historic volatility. \gls{BS} implied volatility is not included in the test conducted in this paper, as consistent with previous research it is assumed that for the implied volatility measures, model-free implied volatility outperforms \gls{BS} implied volatility. \\
Consistent with previous literature, the following hypothesis are tested:\\
\textbf{Hypothesis 1:} The VIX contains information about future realized volatility\\
\textbf{Hypothesis 2:} The VIX is an unbiased estimator of future realized volatility\\
\textbf{Hypothesis 3:} The VIX has more explanatory power than the historical volatility in estimating future realized volatility\\
\textbf{Hypothesis 4:} The VIX incorporates all information regarding future realized volatility, historic volatility contains no information beyond what is already included in the VIX



